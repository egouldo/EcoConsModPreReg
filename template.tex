% Options for packages loaded elsewhere
\PassOptionsToPackage{unicode}{hyperref}
\PassOptionsToPackage{hyphens}{url}
\PassOptionsToPackage{dvipsnames,svgnames,x11names}{xcolor}
%
\documentclass[
]{article}

\usepackage{amsmath,amssymb}
\usepackage{iftex}
\ifPDFTeX
  \usepackage[T1]{fontenc}
  \usepackage[utf8]{inputenc}
  \usepackage{textcomp} % provide euro and other symbols
\else % if luatex or xetex
  \usepackage{unicode-math}
  \defaultfontfeatures{Scale=MatchLowercase}
  \defaultfontfeatures[\rmfamily]{Ligatures=TeX,Scale=1}
\fi
\usepackage{lmodern}
\ifPDFTeX\else  
    % xetex/luatex font selection
\fi
% Use upquote if available, for straight quotes in verbatim environments
\IfFileExists{upquote.sty}{\usepackage{upquote}}{}
\IfFileExists{microtype.sty}{% use microtype if available
  \usepackage[]{microtype}
  \UseMicrotypeSet[protrusion]{basicmath} % disable protrusion for tt fonts
}{}
\makeatletter
\@ifundefined{KOMAClassName}{% if non-KOMA class
  \IfFileExists{parskip.sty}{%
    \usepackage{parskip}
  }{% else
    \setlength{\parindent}{0pt}
    \setlength{\parskip}{6pt plus 2pt minus 1pt}}
}{% if KOMA class
  \KOMAoptions{parskip=half}}
\makeatother
\usepackage{xcolor}
\setlength{\emergencystretch}{3em} % prevent overfull lines
\setcounter{secnumdepth}{5}
% Make \paragraph and \subparagraph free-standing
\makeatletter
\ifx\paragraph\undefined\else
  \let\oldparagraph\paragraph
  \renewcommand{\paragraph}{
    \@ifstar
      \xxxParagraphStar
      \xxxParagraphNoStar
  }
  \newcommand{\xxxParagraphStar}[1]{\oldparagraph*{#1}\mbox{}}
  \newcommand{\xxxParagraphNoStar}[1]{\oldparagraph{#1}\mbox{}}
\fi
\ifx\subparagraph\undefined\else
  \let\oldsubparagraph\subparagraph
  \renewcommand{\subparagraph}{
    \@ifstar
      \xxxSubParagraphStar
      \xxxSubParagraphNoStar
  }
  \newcommand{\xxxSubParagraphStar}[1]{\oldsubparagraph*{#1}\mbox{}}
  \newcommand{\xxxSubParagraphNoStar}[1]{\oldsubparagraph{#1}\mbox{}}
\fi
\makeatother


\providecommand{\tightlist}{%
  \setlength{\itemsep}{0pt}\setlength{\parskip}{0pt}}\usepackage{longtable,booktabs,array}
\usepackage{calc} % for calculating minipage widths
% Correct order of tables after \paragraph or \subparagraph
\usepackage{etoolbox}
\makeatletter
\patchcmd\longtable{\par}{\if@noskipsec\mbox{}\fi\par}{}{}
\makeatother
% Allow footnotes in longtable head/foot
\IfFileExists{footnotehyper.sty}{\usepackage{footnotehyper}}{\usepackage{footnote}}
\makesavenoteenv{longtable}
\usepackage{graphicx}
\makeatletter
\newsavebox\pandoc@box
\newcommand*\pandocbounded[1]{% scales image to fit in text height/width
  \sbox\pandoc@box{#1}%
  \Gscale@div\@tempa{\textheight}{\dimexpr\ht\pandoc@box+\dp\pandoc@box\relax}%
  \Gscale@div\@tempb{\linewidth}{\wd\pandoc@box}%
  \ifdim\@tempb\p@<\@tempa\p@\let\@tempa\@tempb\fi% select the smaller of both
  \ifdim\@tempa\p@<\p@\scalebox{\@tempa}{\usebox\pandoc@box}%
  \else\usebox{\pandoc@box}%
  \fi%
}
% Set default figure placement to htbp
\def\fps@figure{htbp}
\makeatother
% definitions for citeproc citations
\NewDocumentCommand\citeproctext{}{}
\NewDocumentCommand\citeproc{mm}{%
  \begingroup\def\citeproctext{#2}\cite{#1}\endgroup}
\makeatletter
 % allow citations to break across lines
 \let\@cite@ofmt\@firstofone
 % avoid brackets around text for \cite:
 \def\@biblabel#1{}
 \def\@cite#1#2{{#1\if@tempswa , #2\fi}}
\makeatother
\newlength{\cslhangindent}
\setlength{\cslhangindent}{1.5em}
\newlength{\csllabelwidth}
\setlength{\csllabelwidth}{3em}
\newenvironment{CSLReferences}[2] % #1 hanging-indent, #2 entry-spacing
 {\begin{list}{}{%
  \setlength{\itemindent}{0pt}
  \setlength{\leftmargin}{0pt}
  \setlength{\parsep}{0pt}
  % turn on hanging indent if param 1 is 1
  \ifodd #1
   \setlength{\leftmargin}{\cslhangindent}
   \setlength{\itemindent}{-1\cslhangindent}
  \fi
  % set entry spacing
  \setlength{\itemsep}{#2\baselineskip}}}
 {\end{list}}
\usepackage{calc}
\newcommand{\CSLBlock}[1]{\hfill\break\parbox[t]{\linewidth}{\strut\ignorespaces#1\strut}}
\newcommand{\CSLLeftMargin}[1]{\parbox[t]{\csllabelwidth}{\strut#1\strut}}
\newcommand{\CSLRightInline}[1]{\parbox[t]{\linewidth - \csllabelwidth}{\strut#1\strut}}
\newcommand{\CSLIndent}[1]{\hspace{\cslhangindent}#1}

\usepackage{easy-todo}
\usepackage{geometry}
\usepackage{xcolor}
\usepackage{fancyhdr}
\usepackage{sectsty}
\usepackage{fontspec}
\usepackage[noblocks]{authblk}
\renewcommand*{\Authsep}{, }
\renewcommand*{\Authand}{, }
\renewcommand*{\Authands}{, }
\renewcommand\Affilfont{\small}
\usepackage[sfdefault]{cabin}
\usepackage[condensed]{roboto}
\usepackage[T1]{fontenc}
\usepackage{titlesec}
\renewcommand{\abstractname}{Background and Instructions}  %for pdf (renames "abstract" to "Background and Instructions")
\makeatletter
\@ifpackageloaded{tcolorbox}{}{\usepackage[skins,breakable]{tcolorbox}}
\@ifpackageloaded{fontawesome5}{}{\usepackage{fontawesome5}}
\definecolor{quarto-callout-color}{HTML}{909090}
\definecolor{quarto-callout-note-color}{HTML}{0758E5}
\definecolor{quarto-callout-important-color}{HTML}{CC1914}
\definecolor{quarto-callout-warning-color}{HTML}{EB9113}
\definecolor{quarto-callout-tip-color}{HTML}{00A047}
\definecolor{quarto-callout-caution-color}{HTML}{FC5300}
\definecolor{quarto-callout-color-frame}{HTML}{acacac}
\definecolor{quarto-callout-note-color-frame}{HTML}{4582ec}
\definecolor{quarto-callout-important-color-frame}{HTML}{d9534f}
\definecolor{quarto-callout-warning-color-frame}{HTML}{f0ad4e}
\definecolor{quarto-callout-tip-color-frame}{HTML}{02b875}
\definecolor{quarto-callout-caution-color-frame}{HTML}{fd7e14}
\makeatother
\makeatletter
\@ifpackageloaded{caption}{}{\usepackage{caption}}
\AtBeginDocument{%
\ifdefined\contentsname
  \renewcommand*\contentsname{Table of contents}
\else
  \newcommand\contentsname{Table of contents}
\fi
\ifdefined\listfigurename
  \renewcommand*\listfigurename{List of Figures}
\else
  \newcommand\listfigurename{List of Figures}
\fi
\ifdefined\listtablename
  \renewcommand*\listtablename{List of Tables}
\else
  \newcommand\listtablename{List of Tables}
\fi
\ifdefined\figurename
  \renewcommand*\figurename{Figure}
\else
  \newcommand\figurename{Figure}
\fi
\ifdefined\tablename
  \renewcommand*\tablename{Table}
\else
  \newcommand\tablename{Table}
\fi
}
\@ifpackageloaded{float}{}{\usepackage{float}}
\floatstyle{ruled}
\@ifundefined{c@chapter}{\newfloat{codelisting}{h}{lop}}{\newfloat{codelisting}{h}{lop}[chapter]}
\floatname{codelisting}{Listing}
\newcommand*\listoflistings{\listof{codelisting}{List of Listings}}
\makeatother
\makeatletter
\makeatother
\makeatletter
\@ifpackageloaded{caption}{}{\usepackage{caption}}
\@ifpackageloaded{subcaption}{}{\usepackage{subcaption}}
\makeatother

\usepackage{bookmark}

\IfFileExists{xurl.sty}{\usepackage{xurl}}{} % add URL line breaks if available
\urlstyle{same} % disable monospaced font for URLs
\hypersetup{
  pdftitle={Ecological Modelling Preregistration Template},
  pdfkeywords={preregistration, preregistration
template, transparency, reproducibility, open-science},
  colorlinks=true,
  linkcolor={blue},
  filecolor={Maroon},
  citecolor={blue},
  urlcolor={Blue},
  pdfcreator={LaTeX via pandoc}}


% https://stackoverflow.com/a/75043433
\title{Ecological Modelling Preregistration Template}


\author[1]{Elliot Gould}
\author[1]{David Duncan}
\author[1]{Hannah Fraser}
\author[1]{Libby Rumpff}
\author[2]{Jian Yen}
\author[1]{Megan Good}
\author[2]{Chris Jones}
\author[2]{Workshop participants (to be added on acceptance of
co-authorship)}

\affil[1]{University of Melbourne}
\affil[2]{Arthur Rylah Institute for Environmental Research}


\date{2025-03-02}

%% Set page size & margins
\geometry{a4paper, total={170mm,257mm}, left=20mm, top=20mm, bottom=20mm, right=20mm}
\begin{document}
\maketitle
\begin{abstract}
Here we present a preregistration template for ecological models in
ecology, conservation and related fields. For non-trivial modelling
studies, especially where model parameter and structure is in any way
data-contingent, we recommend taking an
\emph{\href{https://egouldo.github.io/EcoConsPreReg/}{Adaptive
Preregistration}} approach (\citeproc{ref-Gould2024a}{Gould et al.,
2024}).

Replace author, author-affiliations and persistent ID's
(e.g.~\href{https://orcid.org/}{ORCID iD}), keywords, title and abstract
metadata as relevant to your study.

All preregistration items should be completed, excluding items marked as
optional or in cases where they are not applicable to your study.
Additional preregistration items can be added as required at the
researchers' discretion.
\end{abstract}

\renewcommand*\contentsname{Table of contents}
{
\hypersetup{linkcolor=}
\setcounter{tocdepth}{2}
\tableofcontents
}

\section*{Study Information}\label{study-information}
\addcontentsline{toc}{section}{Study Information}

\subsection*{CRediT Contribution
Statement}\label{credit-contribution-statement}
\addcontentsline{toc}{subsection}{CRediT Contribution Statement}

\begin{tcolorbox}[enhanced jigsaw, rightrule=.15mm, titlerule=0mm, coltitle=black, opacityback=0, bottomrule=.15mm, colback=white, opacitybacktitle=0.6, title=\textcolor{quarto-callout-caution-color}{\faFire}\hspace{0.5em}{Preregistration Item}, toprule=.15mm, colframe=quarto-callout-caution-color-frame, left=2mm, leftrule=.75mm, breakable, bottomtitle=1mm, colbacktitle=quarto-callout-caution-color!10!white, toptitle=1mm, arc=.35mm]

Identify potential contributions according to the CRediT taxonomy
(\url{https://doi.org/10.1371/journal.pone.0244611.t001}) and write a
\href{https://authorservices.wiley.com/author-resources/Journal-Authors/open-access/credit.html}{CRediT
contribution statement}.

\end{tcolorbox}

\subsection*{Conflict of Interest
Statement}\label{conflict-of-interest-statement}
\addcontentsline{toc}{subsection}{Conflict of Interest Statement}

\begin{tcolorbox}[enhanced jigsaw, rightrule=.15mm, titlerule=0mm, coltitle=black, opacityback=0, bottomrule=.15mm, colback=white, opacitybacktitle=0.6, title=\textcolor{quarto-callout-caution-color}{\faFire}\hspace{0.5em}{Preregistration Item}, toprule=.15mm, colframe=quarto-callout-caution-color-frame, left=2mm, leftrule=.75mm, breakable, bottomtitle=1mm, colbacktitle=quarto-callout-caution-color!10!white, toptitle=1mm, arc=.35mm]

\begin{itemize}
\tightlist
\item[$\square$]
  Explain any real or perceived conflicts of interest with this study
  execution. For example, any interests or activities that might be seen
  as influencing the research (e.g., financial interests in a test or
  procedure, funding by companies for research).
\end{itemize}

\end{tcolorbox}

\subsection*{Data Availability
Statement}\label{data-availability-statement}
\addcontentsline{toc}{subsection}{Data Availability Statement}

\begin{tcolorbox}[enhanced jigsaw, rightrule=.15mm, titlerule=0mm, coltitle=black, opacityback=0, bottomrule=.15mm, colback=white, opacitybacktitle=0.6, title=\textcolor{quarto-callout-caution-color}{\faFire}\hspace{0.5em}{Preregistration Item}, toprule=.15mm, colframe=quarto-callout-caution-color-frame, left=2mm, leftrule=.75mm, breakable, bottomtitle=1mm, colbacktitle=quarto-callout-caution-color!10!white, toptitle=1mm, arc=.35mm]

\emph{Select one option from below:}

\begin{itemize}
\tightlist
\item[$\square$]
  ``We plan to make the data available (yes / no),'' specify the planned
  data availability level from the following options:

  \begin{itemize}
  \tightlist
  \item
    Data access via download; usage of data for all purposes (public use
    file)
  \item
    Data access via download; usage of data restricted to scientific
    purposes (scientific use file)
  \item
    Data access via download; usage of data has to be agreed and defined
    on an individual case basis
  \item
    Data access via secure data center (no download, usage/analysis only
    in a secure data center)
  \item
    Data available upon email request by member of scientific community
  \item
    Other (please specify)
  \end{itemize}
\item[$\square$]
  ``Data will not be made available''

  \begin{itemize}
  \tightlist
  \item[$\square$]
    Justify reason for not making data available.
  \end{itemize}
\end{itemize}

\end{tcolorbox}

\subsection*{Code Availability}\label{code-availability}
\addcontentsline{toc}{subsection}{Code Availability}

\begin{tcolorbox}[enhanced jigsaw, rightrule=.15mm, titlerule=0mm, coltitle=black, opacityback=0, bottomrule=.15mm, colback=white, opacitybacktitle=0.6, title=\textcolor{quarto-callout-caution-color}{\faFire}\hspace{0.5em}{Preregistration Item}, toprule=.15mm, colframe=quarto-callout-caution-color-frame, left=2mm, leftrule=.75mm, breakable, bottomtitle=1mm, colbacktitle=quarto-callout-caution-color!10!white, toptitle=1mm, arc=.35mm]

\emph{Select one option from below:}

\begin{itemize}
\tightlist
\item[$\square$]
  ``We plan to make the code available (yes / no),'' specify the planned
  code availability level from the following options:

  \begin{itemize}
  \tightlist
  \item
    Code access via download; usage of code for all purposes (public use
    file)
  \item
    Code access via download; usage of code restricted to scientific
    purposes (scientific use file)
  \item
    Code access via download; usage of code has to be agreed and defined
    on an individual case basis
  \item
    Code access via secure code center (no download, usage/analysis only
    in a secure code center)
  \item
    Code available upon email request by member of scientific community
  \item
    Other (please specify)
  \end{itemize}
\item[$\square$]
  ``Code will not be made available''

  \begin{itemize}
  \tightlist
  \item[$\square$]
    Justify reason for not making code available.
  \end{itemize}
\end{itemize}

\end{tcolorbox}

\subsection*{Ethics}\label{ethics}
\addcontentsline{toc}{subsection}{Ethics}

\begin{tcolorbox}[enhanced jigsaw, rightrule=.15mm, titlerule=0mm, coltitle=black, opacityback=0, bottomrule=.15mm, colback=white, opacitybacktitle=0.6, title=\textcolor{quarto-callout-caution-color}{\faFire}\hspace{0.5em}{Preregistration Item}, toprule=.15mm, colframe=quarto-callout-caution-color-frame, left=2mm, leftrule=.75mm, breakable, bottomtitle=1mm, colbacktitle=quarto-callout-caution-color!10!white, toptitle=1mm, arc=.35mm]

\begin{itemize}
\tightlist
\item[$\square$]
  Select and respond to the relevant item below:

  \begin{itemize}
  \tightlist
  \item
    If relevant institutional ethical approval for the study has been
    obtained, provide the relevant identifier, and link to relevant
    documents.
  \item
    If ethical approval has not yet been obtained, but is required,
    provide a brief overview of plans for obtaining study approval in
    accordance with established ethical guidelines.
  \item
    Alternatively, if the study is exempt from ethical approval, explain
    exemption.
  \end{itemize}
\end{itemize}

\end{tcolorbox}

\section{Problem Formulation}\label{sec-problem-formulation}

\begin{tcolorbox}[enhanced jigsaw, rightrule=.15mm, titlerule=0mm, coltitle=black, opacityback=0, bottomrule=.15mm, colback=white, opacitybacktitle=0.6, title=\textcolor{quarto-callout-note-color}{\faInfo}\hspace{0.5em}{Rationale \& Explanation}, toprule=.15mm, colframe=quarto-callout-note-color-frame, left=2mm, leftrule=.75mm, breakable, bottomtitle=1mm, colbacktitle=quarto-callout-note-color!10!white, toptitle=1mm, arc=.35mm]

This section specifies the decision-making context in which the model
will be used or the intended scope and context of conclusions. Important
components include the decision maker and stakeholders (including
experts) and their view on: i) the nature of the problem or decision
addressed and how the scope of the modelling tool fits within the
(broader) context (i.e.~model purpose; ii) the spatial and temporal
scales relevant to the decision context; iii) specified desired outputs;
iv) role and inclusion in model development and testing; v) whether they
foresee unacceptable outcomes that need to be represented in the model
(i.e.~as constraints), and; vi) what future scenarios does the model
need to account for (noting this may be revised later). It should also
provide a summary of the domain of applicability of the model, and
reasonable extrapolation limits (\citeproc{ref-Grimm:2014es}{Grimm et
al., 2014}).

\end{tcolorbox}

\subsection{Model Context and Purpose}\label{model-context-and-purpose}

\begin{tcolorbox}[enhanced jigsaw, rightrule=.15mm, titlerule=0mm, coltitle=black, opacityback=0, bottomrule=.15mm, colback=white, opacitybacktitle=0.6, title=\textcolor{quarto-callout-note-color}{\faInfo}\hspace{0.5em}{Rationale \& Explanation}, toprule=.15mm, colframe=quarto-callout-note-color-frame, left=2mm, leftrule=.75mm, breakable, bottomtitle=1mm, colbacktitle=quarto-callout-note-color!10!white, toptitle=1mm, arc=.35mm]

Defining the purpose of the model is critical because the model purpose
influences choices at later stages of model development
(\citeproc{ref-Jakeman:2006ii}{Jakeman et al., 2006}). Common model
purposes in ecology include: gaining a better qualitative understanding
of the target system, synthesising and reviewing knowledge, and
providing guidance for management and decision-making
(\citeproc{ref-Jakeman:2006ii}{Jakeman et al., 2006}). Note that
modelling objectives are distinct from the analytical objectives of the
model.

The scope of the model includes temporal and spatial resolutions, which
should also be defined here (\citeproc{ref-Mahmoud2009}{Mahmoud et al.,
2009}). Any external limitations on model development, analysis and
flexibility should also be outlined in this section
(\citeproc{ref-Jakeman:2006ii}{Jakeman et al., 2006}).

\end{tcolorbox}

\subsubsection{Key stakeholders and model
users}\label{key-stakeholders-and-model-users}

\begin{tcolorbox}[enhanced jigsaw, rightrule=.15mm, titlerule=0mm, coltitle=black, opacityback=0, bottomrule=.15mm, colback=white, opacitybacktitle=0.6, title=\textcolor{quarto-callout-caution-color}{\faFire}\hspace{0.5em}{Preregistration Item}, toprule=.15mm, colframe=quarto-callout-caution-color-frame, left=2mm, leftrule=.75mm, breakable, bottomtitle=1mm, colbacktitle=quarto-callout-caution-color!10!white, toptitle=1mm, arc=.35mm]

\emph{Identify relevant interest groups:}

\begin{itemize}
\tightlist
\item[$\square$]
  Who is the model for?
\item[$\square$]
  Who is involved in formulating the model?
\item[$\square$]
  How will key stakeholders be involved in model development?
\item[$\square$]
  Describe the decision-making context in which the model will be used
  (if relevant).
\end{itemize}

\end{tcolorbox}

\subsubsection{Model purpose, context and problem
context}\label{sec-model-purpose-context-and-problem-context}

\begin{tcolorbox}[enhanced jigsaw, rightrule=.15mm, titlerule=0mm, coltitle=black, opacityback=0, bottomrule=.15mm, colback=white, opacitybacktitle=0.6, title=\textcolor{quarto-callout-caution-color}{\faFire}\hspace{0.5em}{Preregistration Item}, toprule=.15mm, colframe=quarto-callout-caution-color-frame, left=2mm, leftrule=.75mm, breakable, bottomtitle=1mm, colbacktitle=quarto-callout-caution-color!10!white, toptitle=1mm, arc=.35mm]

\emph{Briefly outline:}

\begin{itemize}
\tightlist
\item[$\square$]
  the ecological problem,
\item[$\square$]
  the decision problem (if relevant), including the decision-trigger and
  any regulatory frameworks relevant to the problem,
\item[$\square$]
  how the model will address the problem, being clear about the scope of
  the model i.e.~is the model addressing the whole problem, or part of
  it? Are there any linked problems that your model should consider?
\item[$\square$]
  Ensure that you specify any focal taxa and study objectives.
\end{itemize}

\end{tcolorbox}

\subsubsection{Analytical objectives}\label{sec-analytical-objectives}

\begin{tcolorbox}[enhanced jigsaw, rightrule=.15mm, titlerule=0mm, coltitle=black, opacityback=0, bottomrule=.15mm, colback=white, opacitybacktitle=0.6, title=\textcolor{quarto-callout-note-color}{\faInfo}\hspace{0.5em}{Explanation}, toprule=.15mm, colframe=quarto-callout-note-color-frame, left=2mm, leftrule=.75mm, breakable, bottomtitle=1mm, colbacktitle=quarto-callout-note-color!10!white, toptitle=1mm, arc=.35mm]

How will the model be analysed, what analytical questions will the model
be used to answer? For example, you might be using your model in a
scenario analysis to determine which management decision is associated
with minimum regret or the highest likelihood of improvement. Other
examples from ecological decision-making include: to compare the
performance of alternative management actions under budget constraint
(\citeproc{ref-Fraser:2017jf}{Fraser et al., 2017}), to search for
robust decisions under uncertainty
(\citeproc{ref-McDonald-Madden2008}{McDonald-Madden et al., 2008}), to
choose the conservation policy that minimises uncertainty
(\citeproc{ref-McCarthy2011}{McCarthy et al., 2011}). See other examples
in (\citeproc{ref-Moallemi2019}{Moallemi et al., 2019}).

\end{tcolorbox}

\begin{tcolorbox}[enhanced jigsaw, rightrule=.15mm, titlerule=0mm, coltitle=black, opacityback=0, bottomrule=.15mm, colback=white, opacitybacktitle=0.6, title=\textcolor{quarto-callout-caution-color}{\faFire}\hspace{0.5em}{Preregistration Item}, toprule=.15mm, colframe=quarto-callout-caution-color-frame, left=2mm, leftrule=.75mm, breakable, bottomtitle=1mm, colbacktitle=quarto-callout-caution-color!10!white, toptitle=1mm, arc=.35mm]

\emph{Provide detail on the analytical purpose and scope of the model:}

\begin{itemize}
\tightlist
\item[$\square$]
  How will the model be analysed and what analytical questions will the
  model be used to answer?
\item[$\square$]
  Candidate decisions should be investigated and are specified a priori.
  Depending on the modelling context, they may be specified by
  stakeholders, model users or the analyst
  (\citeproc{ref-Moallemi2019}{Moallemi et al., 2019}).

  \begin{itemize}
  \tightlist
  \item[$\square$]
    Describe the method used to identify relevant management actions and
  \item[$\square$]
    specify management actions to be considered included in the model.
  \item[$\square$]
    Are there potentially unacceptable management or policy outcomes
    identified by stakeholders that should be captured in the model,
    i.e.~as constraints?
  \end{itemize}
\item[$\square$]
  Are there scenarios that model inputs or outputs that must
  accommodated? Scenarios should be set a priori, (i.e.~before the model
  is built, \citeproc{ref-Moallemi2019}{Moallemi et al., 2019}) and may
  be stakeholder-defined or driven by the judgement of the modeller or
  other experts (\citeproc{ref-Mahmoud2009}{Mahmoud et al., 2009}).

  \begin{itemize}
  \tightlist
  \item[$\square$]
    If relevant, describe what processes you will use to elicit and
    identify relevant scenarios, e.g.~literature review, structured
    workshops with stakeholders or decision-makers.
  \item[$\square$]
    Specify scenarios under which decisions are investigated.
  \end{itemize}
\end{itemize}

\end{tcolorbox}

\subsubsection{Logistical Constraints}\label{logistical-constraints}

\begin{tcolorbox}[enhanced jigsaw, rightrule=.15mm, titlerule=0mm, coltitle=black, opacityback=0, bottomrule=.15mm, colback=white, opacitybacktitle=0.6, title=\textcolor{quarto-callout-caution-color}{\faFire}\hspace{0.5em}{Preregistration Item}, toprule=.15mm, colframe=quarto-callout-caution-color-frame, left=2mm, leftrule=.75mm, breakable, bottomtitle=1mm, colbacktitle=quarto-callout-caution-color!10!white, toptitle=1mm, arc=.35mm]

\begin{itemize}
\tightlist
\item[$\square$]
  What degree of flexibility is required from the model? Might the model
  need to be quickly reconfigured to explore new scenarios or problems
  proposed by clients / managers / model-users?
\item[$\square$]
  Are there any limitations on model development analysis and
  flexibility, such as time or budget constraints, for example, does a
  model need to be deployed rapidly?

  \begin{itemize}
  \tightlist
  \item[$\square$]
    When must the model be completed by, e.g.~to help make a decision?
  \end{itemize}
\end{itemize}

\end{tcolorbox}

\subsubsection{Model Scope, Scale and
Resolution}\label{model-scope-scale-and-resolution}

\begin{tcolorbox}[enhanced jigsaw, rightrule=.15mm, titlerule=0mm, coltitle=black, opacityback=0, bottomrule=.15mm, colback=white, opacitybacktitle=0.6, title=\textcolor{quarto-callout-caution-color}{\faFire}\hspace{0.5em}{Preregistration Item}, toprule=.15mm, colframe=quarto-callout-caution-color-frame, left=2mm, leftrule=.75mm, breakable, bottomtitle=1mm, colbacktitle=quarto-callout-caution-color!10!white, toptitle=1mm, arc=.35mm]

\begin{itemize}
\tightlist
\item[$\square$]
  The choice of a model's boundaries is closely linked to the choice of
  how finely to aggregate the behaviour within the model
  (\citeproc{ref-Jakeman:2006ii}{Jakeman et al., 2006}) - what is the
  intended scale, and resolution of the model (temporal, spatial or
  otherwise)?
\item[$\square$]
  Where is the boundary of the modelled system? Everything outside
  beyond the boundary and not crossing it is to be ignored within the
  domain of the model, and everything crossing the boundary is to be
  treated as external forcing (known/unknown), or else as model outputs
  (observed, or not, \citeproc{ref-Jakeman:2006ii}{Jakeman et al.,
  2006}).
\end{itemize}

\end{tcolorbox}

\subsubsection{Intended application of
results}\label{intended-application-of-results}

\begin{tcolorbox}[enhanced jigsaw, rightrule=.15mm, titlerule=0mm, coltitle=black, opacityback=0, bottomrule=.15mm, colback=white, opacitybacktitle=0.6, title=\textcolor{quarto-callout-note-color}{\faInfo}\hspace{0.5em}{Explanation}, toprule=.15mm, colframe=quarto-callout-note-color-frame, left=2mm, leftrule=.75mm, breakable, bottomtitle=1mm, colbacktitle=quarto-callout-note-color!10!white, toptitle=1mm, arc=.35mm]

Preregistration Items in this section are relevant to model
transferability (\citeproc{ref-Yates2018}{Yates et al., 2018}) and
constraints on generality in model analysis interpretation. How far do
can the results be extrapolated based on the study design (data + model
+ analysis)? For instance, if there are many confounding variables and
not enough spatial / environmental replication, then making broader more
general claims beyond the stated boundaries of the model
(Section~\ref{sec-analytical-objectives}) may not be warranted. However,
larger generalisations about results may be acceptable if the data comes
from experimentally manipulated or controlled systems.

\end{tcolorbox}

\begin{tcolorbox}[enhanced jigsaw, rightrule=.15mm, titlerule=0mm, coltitle=black, opacityback=0, bottomrule=.15mm, colback=white, opacitybacktitle=0.6, title=\textcolor{quarto-callout-caution-color}{\faFire}\hspace{0.5em}{Preregistration Item}, toprule=.15mm, colframe=quarto-callout-caution-color-frame, left=2mm, leftrule=.75mm, breakable, bottomtitle=1mm, colbacktitle=quarto-callout-caution-color!10!white, toptitle=1mm, arc=.35mm]

\begin{itemize}
\tightlist
\item[$\square$]
  What is the intended domain in which the model is to be applied? Are
  there any reasonable extrapolation limits beyond which you expect the
  model should not be applied (\citeproc{ref-Grimm:2014es}{Grimm et al.,
  2014})?
\end{itemize}

\end{tcolorbox}

\subsection{Scenario Analysis
Operationalisation}\label{scenario-analysis-operationalisation}

\begin{tcolorbox}[enhanced jigsaw, rightrule=.15mm, titlerule=0mm, coltitle=black, opacityback=0, bottomrule=.15mm, colback=white, opacitybacktitle=0.6, title=\textcolor{quarto-callout-caution-color}{\faFire}\hspace{0.5em}{Preregistration Item (delete as necessary)}, toprule=.15mm, colframe=quarto-callout-caution-color-frame, left=2mm, leftrule=.75mm, breakable, bottomtitle=1mm, colbacktitle=quarto-callout-caution-color!10!white, toptitle=1mm, arc=.35mm]

\begin{itemize}
\tightlist
\item[$\square$]
  How will you operationalise any scenarios identified in
  Section~\ref{sec-analytical-objectives}? For example, how will you
  operationalise any qualitative changes of interest, such as ‚
  `deterioration' or `improvement'?
\item[$\square$]
  Describe how you will evaluate and distinguish the performance of
  alternative scenario outcomes
\item[$\square$]
  Justify or otherwise explain how you chose these measures and
  determined performance criteria in relation to the analytical
  objectives, model purpose and modelling context, such as the risk
  attitudes of decision-makers and stakeholders within this system
\end{itemize}

\end{tcolorbox}

\section{Define Conceptual Model}\label{define-conceptual-model}

\begin{tcolorbox}[enhanced jigsaw, rightrule=.15mm, titlerule=0mm, coltitle=black, opacityback=0, bottomrule=.15mm, colback=white, opacitybacktitle=0.6, title=\textcolor{quarto-callout-note-color}{\faInfo}\hspace{0.5em}{Explanation}, toprule=.15mm, colframe=quarto-callout-note-color-frame, left=2mm, leftrule=.75mm, breakable, bottomtitle=1mm, colbacktitle=quarto-callout-note-color!10!white, toptitle=1mm, arc=.35mm]

Conceptual models underpin the formal or quantitative model
(\citeproc{ref-Cartwright:2016kr}{Cartwright et al., 2016}). The
conceptual model describes the biological mechanisms relevant to the
ecological problem and should capture basic premises about how the
target system works, including any prior knowledge and assumptions about
system processes. Conceptual models may be represented in a variety of
formats, such as influence diagrams, linguistic model block diagram or
bond graphs, and these illustrate how model drivers are linked to both
outputs or observed responses, and internal (state) variables
(\citeproc{ref-Jakeman:2006ii}{Jakeman et al., 2006}).

\end{tcolorbox}

\subsection{Choose elicitation and representation
method}\label{choose-elicitation-and-representation-method}

\begin{tcolorbox}[enhanced jigsaw, rightrule=.15mm, titlerule=0mm, coltitle=black, opacityback=0, bottomrule=.15mm, colback=white, opacitybacktitle=0.6, title=\textcolor{quarto-callout-caution-color}{\faFire}\hspace{0.5em}{Preregistration Item}, toprule=.15mm, colframe=quarto-callout-caution-color-frame, left=2mm, leftrule=.75mm, breakable, bottomtitle=1mm, colbacktitle=quarto-callout-caution-color!10!white, toptitle=1mm, arc=.35mm]

\begin{itemize}
\tightlist
\item[$\square$]
  Describe what method you will use to elicit or identify the conceptual
  model. Some common methods include interviews, drawings, and mapping
  techniques including influence diagrams, cognitive maps and Bayesian
  belief networks (\citeproc{ref-Moon2019}{Moon et al., 2019}). It is
  difficult to decide and justify which method is most appropriate, see
  Moon et al. (\citeproc{ref-Moon2019}{2019}) for guidance addressing
  this methodological question.
\item[$\square$]
  Finally, how do you intend on representing the final conceptual model?
  This will likely depend on the method chosen to elicit the conceptual
  model.
\end{itemize}

\end{tcolorbox}

\subsection{Explain Critical Conceptual Design
Decisions}\label{explain-critical-conceptual-design-decisions}

\begin{tcolorbox}[enhanced jigsaw, rightrule=.15mm, titlerule=0mm, coltitle=black, opacityback=0, bottomrule=.15mm, colback=white, opacitybacktitle=0.6, title=\textcolor{quarto-callout-caution-color}{\faFire}\hspace{0.5em}{Preregistration Item}, toprule=.15mm, colframe=quarto-callout-caution-color-frame, left=2mm, leftrule=.75mm, breakable, bottomtitle=1mm, colbacktitle=quarto-callout-caution-color!10!white, toptitle=1mm, arc=.35mm]

\emph{List and explain critical conceptual design decisions
(\citeproc{ref-Grimm:2014es}{Grimm et al., 2014}), including:}

\begin{itemize}
\tightlist
\item[$\square$]
  spatial and temporal scales,
\item[$\square$]
  selection of entities and processes,
\item[$\square$]
  representation of stochasticity and heterogeneity,
\item[$\square$]
  consideration of local versus global interactions, environmental
  drivers, etc.
\item[$\square$]
  Explain and justify the influence of particular theories, concepts, or
  earlier models against alternative conceptual design decisions that
  might lead to alternative model structures.
\end{itemize}

\end{tcolorbox}

\subsection{Model assumptions and
uncertainties}\label{model-assumptions-and-uncertainties}

\begin{tcolorbox}[enhanced jigsaw, rightrule=.15mm, titlerule=0mm, coltitle=black, opacityback=0, bottomrule=.15mm, colback=white, opacitybacktitle=0.6, title=\textcolor{quarto-callout-caution-color}{\faFire}\hspace{0.5em}{Preregistration Item}, toprule=.15mm, colframe=quarto-callout-caution-color-frame, left=2mm, leftrule=.75mm, breakable, bottomtitle=1mm, colbacktitle=quarto-callout-caution-color!10!white, toptitle=1mm, arc=.35mm]

\emph{Specify key assumptions and uncertainties underlying the model
design, describing how uncertainty and variation will be represented in
the model (\citeproc{ref-Moallemi2019}{Moallemi et al., 2019}). Sources
of uncertainty may include:}

\begin{itemize}
\tightlist
\item[$\square$]
  exogenous uncertainties affecting the system,
\item[$\square$]
  parametric uncertainty in input data and
\item[$\square$]
  structural / conceptual nonparametric uncertainty in the model.
\end{itemize}

\end{tcolorbox}

\subsection{Identify predictor and response
variables}\label{identify-predictor-and-response-variables}

\begin{tcolorbox}[enhanced jigsaw, rightrule=.15mm, titlerule=0mm, coltitle=black, opacityback=0, bottomrule=.15mm, colback=white, opacitybacktitle=0.6, title=\textcolor{quarto-callout-note-color}{\faInfo}\hspace{0.5em}{Explanation}, toprule=.15mm, colframe=quarto-callout-note-color-frame, left=2mm, leftrule=.75mm, breakable, bottomtitle=1mm, colbacktitle=quarto-callout-note-color!10!white, toptitle=1mm, arc=.35mm]

The identification and definition of primary model input variables
should be driven by scenario definitions, and by the scope of the model
described in the problem formulation phase
(\citeproc{ref-Mahmoud2009}{Mahmoud et al., 2009}).

\end{tcolorbox}

\begin{tcolorbox}[enhanced jigsaw, rightrule=.15mm, titlerule=0mm, coltitle=black, opacityback=0, bottomrule=.15mm, colback=white, opacitybacktitle=0.6, title=\textcolor{quarto-callout-caution-color}{\faFire}\hspace{0.5em}{Preregistration Item}, toprule=.15mm, colframe=quarto-callout-caution-color-frame, left=2mm, leftrule=.75mm, breakable, bottomtitle=1mm, colbacktitle=quarto-callout-caution-color!10!white, toptitle=1mm, arc=.35mm]

\emph{Identify and define system system variables and structures,
referencing scenario definitions, and the scope of the model as
described within problem formulation:}

\begin{itemize}
\tightlist
\item[$\square$]
  What variables would support taking this action or making this
  decision?
\item[$\square$]
  What additional variables may interact with this system (things we
  can't control, but can hopefully measure)?
\item[$\square$]
  What variables have not been measured, but may interact with the
  system (often occurs in field or observational studies)?
\item[$\square$]
  What variables are index or surrogate measures of variables that we
  cannot or have not measured?
\item[$\square$]
  In what ways do we expect these variables to interact (model
  structures)?
\item[$\square$]
  Explain how any key concepts or terms within problem or
  decision-making contexts, such as regulatory terms, will be
  operationalised and defined in a biologically meaningful way to answer
  the research question appropriately?
\end{itemize}

\end{tcolorbox}

\subsection{Define prior knowledge, data specification and
evaluation}\label{define-prior-knowledge-data-specification-and-evaluation}

\begin{tcolorbox}[enhanced jigsaw, rightrule=.15mm, titlerule=0mm, coltitle=black, opacityback=0, bottomrule=.15mm, colback=white, opacitybacktitle=0.6, title=\textcolor{quarto-callout-note-color}{\faInfo}\hspace{0.5em}{Explanation}, toprule=.15mm, colframe=quarto-callout-note-color-frame, left=2mm, leftrule=.75mm, breakable, bottomtitle=1mm, colbacktitle=quarto-callout-note-color!10!white, toptitle=1mm, arc=.35mm]

This section specifies the plan for collecting, processing and preparing
data available for parameterisation, determining model structure, and
for scenario analysis. It also allows the researchers to disclose any
prior interaction with the data.

\end{tcolorbox}

\subsubsection{Collate available data sources that could be used to
parameterise or structure the
model}\label{collate-available-data-sources-that-could-be-used-to-parameterise-or-structure-the-model}

\begin{tcolorbox}[enhanced jigsaw, rightrule=.15mm, titlerule=0mm, coltitle=black, opacityback=0, bottomrule=.15mm, colback=white, opacitybacktitle=0.6, title=\textcolor{quarto-callout-caution-color}{\faFire}\hspace{0.5em}{Preregistration Item}, toprule=.15mm, colframe=quarto-callout-caution-color-frame, left=2mm, leftrule=.75mm, breakable, bottomtitle=1mm, colbacktitle=quarto-callout-caution-color!10!white, toptitle=1mm, arc=.35mm]

\textbf{For pre-existing data (delete as appropriate):}

\begin{itemize}
\tightlist
\item[$\square$]
  Document the identity, quantity and provenance of any data that will
  be used to develop, identify and test the model.
\item[$\square$]
  For each dataset, is the data open or publicly available?
\item[$\square$]
  How can the data be accessed? Provide a link or contact as
  appropriate, indicating any restrictions on the use of data.
\item[$\square$]
  Date of download, access, or expected timing offuture access.
\item[$\square$]
  Describe the source of the data - what entity originally collected
  this data? (National Data Set, Private Organisational Data, Own Lab
  Collection, Other Lab Collection, External Contractor, Meta-Analysis,
  Expert Elicitation, Other).
\item[$\square$]
  Codebook and meta-data. If a codebook or other meta-data is available,
  link to it here and / or upload the document(s).
\item[$\square$]
  Prior work based on this dataset - Have you published / presented any
  previous work based on this dataset? Include any publications,
  conference presentations (papers, posters), or working papers
  (in-prep, unpublished, preprints) based on this dataset you have
  worked on.
\item[$\square$]
  Unpublished Prior Research Activity - Describe any prior but
  unpublished research activity using these data. Be specific and
  transparent.
\item[$\square$]
  Prior knowledge of the current dataset - Describe any prior knowledge
  of or interaction with the dataset before commencing this study. For
  example, have you read any reports or publications about this data?
\item[$\square$]
  Describe how the data is arranged, in terms of replicates and
  covariates.
\end{itemize}

\textbf{Sampling Plan (for data you will collect, delete as
appropriate):}

\begin{itemize}
\tightlist
\item[$\square$]
  Data collection procedures - Please describe your data collection
  process, including how sites and transects or any other physical unit
  were selected and arranged. Describe any inclusion or exclusion rules,
  and the study timeline.
\item[$\square$]
  Sample Size - Describe the sample size of your study.
\item[$\square$]
  Sample Size Rationale - Describe how you determined the appropriate
  sample size for your study. It could include feasibility constraints,
  such as time, money or personnel.
\item[$\square$]
  If sample size cannot be specified, specify a stopping rule - i.e.~how
  will you decide when to terminate your data collection?
\end{itemize}

\end{tcolorbox}

\subsubsection{Data Processing and
Preparation}\label{data-processing-and-preparation}

\begin{tcolorbox}[enhanced jigsaw, rightrule=.15mm, titlerule=0mm, coltitle=black, opacityback=0, bottomrule=.15mm, colback=white, opacitybacktitle=0.6, title=\textcolor{quarto-callout-caution-color}{\faFire}\hspace{0.5em}{Preregistration Item}, toprule=.15mm, colframe=quarto-callout-caution-color-frame, left=2mm, leftrule=.75mm, breakable, bottomtitle=1mm, colbacktitle=quarto-callout-caution-color!10!white, toptitle=1mm, arc=.35mm]

\begin{itemize}
\tightlist
\item[$\square$]
  Describe any data preparation and processing steps, including
  manipulation of environmental layers (e.g.~standardisation and
  geographic projection) or variable construction (e.g.~Principal
  Component Analysis).
\end{itemize}

\end{tcolorbox}

\subsubsection{Describe any data exploration or preliminary data
analyses.}\label{describe-any-data-exploration-or-preliminary-data-analyses.}

\begin{tcolorbox}[enhanced jigsaw, rightrule=.15mm, titlerule=0mm, coltitle=black, opacityback=0, bottomrule=.15mm, colback=white, opacitybacktitle=0.6, title=\textcolor{quarto-callout-note-color}{\faInfo}\hspace{0.5em}{Explanation}, toprule=.15mm, colframe=quarto-callout-note-color-frame, left=2mm, leftrule=.75mm, breakable, bottomtitle=1mm, colbacktitle=quarto-callout-note-color!10!white, toptitle=1mm, arc=.35mm]

In most modelling cases, it is necessary to perform preliminary analyses
to understand the data and check that assumptions and requirements of
the chosen modelling procedures are met. Data exploration prior to model
fitting or development may include exploratory analyses to check for
collinearity, spatial and temporal coverage, quality and resolution,
outliers, or the need for transformations
(\citeproc{ref-Yates2018}{Yates et al., 2018}).

\end{tcolorbox}

\begin{tcolorbox}[enhanced jigsaw, rightrule=.15mm, titlerule=0mm, coltitle=black, opacityback=0, bottomrule=.15mm, colback=white, opacitybacktitle=0.6, title=\textcolor{quarto-callout-caution-color}{\faFire}\hspace{0.5em}{Preregistration Item}, toprule=.15mm, colframe=quarto-callout-caution-color-frame, left=2mm, leftrule=.75mm, breakable, bottomtitle=1mm, colbacktitle=quarto-callout-caution-color!10!white, toptitle=1mm, arc=.35mm]

\emph{For each separate preliminary or investigatory analysis:}

\begin{itemize}
\tightlist
\item[$\square$]
  State what needs to be known to proceed with further decision-making
  about the modelling procedure, and why the analysis is necessary.
\item[$\square$]
  Explain how you will implement this analysis, as well as any
  techniques you will use to summarise and explore your data.
\item[$\square$]
  What method will you use to represent this analysis (graphical,
  tabular, or otherwise, describe)
\item[$\square$]
  Specify exactly which parts of the data will be used
\item[$\square$]
  Describe how the results will be interpreted, listing each potential
  analytic decision, as well as the analysis finding that will trigger
  each decision, where possible.
\end{itemize}

\end{tcolorbox}

\subsubsection{Data evaluation, exclusion and missing
data}\label{data-evaluation-exclusion-and-missing-data}

\begin{tcolorbox}[enhanced jigsaw, rightrule=.15mm, titlerule=0mm, coltitle=black, opacityback=0, bottomrule=.15mm, colback=white, opacitybacktitle=0.6, title=\textcolor{quarto-callout-note-color}{\faInfo}\hspace{0.5em}{Explanation}, toprule=.15mm, colframe=quarto-callout-note-color-frame, left=2mm, leftrule=.75mm, breakable, bottomtitle=1mm, colbacktitle=quarto-callout-note-color!10!white, toptitle=1mm, arc=.35mm]

Documenting issues with reliability is important because data quality
and ecological relevance might be constrained by measurement error,
inappropriate experimental design, and heterogeneity and variability
inherent in ecological systems (\citeproc{ref-Grimm:2014es}{Grimm et
al., 2014}). Ideally, model input data should be internally consistent
across temporal and spatial scales and resolutions, and appropriate to
the problem at hand (\citeproc{ref-Mahmoud2009}{Mahmoud et al., 2009}).

\end{tcolorbox}

\begin{tcolorbox}[enhanced jigsaw, rightrule=.15mm, titlerule=0mm, coltitle=black, opacityback=0, bottomrule=.15mm, colback=white, opacitybacktitle=0.6, title=\textcolor{quarto-callout-caution-color}{\faFire}\hspace{0.5em}{Preregistration Item}, toprule=.15mm, colframe=quarto-callout-caution-color-frame, left=2mm, leftrule=.75mm, breakable, bottomtitle=1mm, colbacktitle=quarto-callout-caution-color!10!white, toptitle=1mm, arc=.35mm]

\begin{itemize}
\tightlist
\item[$\square$]
  Describe how you will determine how reliable the data is for the given
  model purpose. Ideally, model input data should be internally
  consistent across temporal and spatial scales and resolutions, and
  appropriate to the problem at hand
\item[$\square$]
  Document any issues with data reliability.
\item[$\square$]
  How will you determine what data, if any, will be excluded from your
  analyses?
\item[$\square$]
  How will outliers be handled? Describe rules for identifying outlier
  data, and for excluding a site, transect, quadrat, year or season,
  species, trait, etc.
\item[$\square$]
  How will you identify and deal with incomplete or missing data?
\end{itemize}

\end{tcolorbox}

\subsection{Conceptual model
evaluation}\label{conceptual-model-evaluation}

\begin{tcolorbox}[enhanced jigsaw, rightrule=.15mm, titlerule=0mm, coltitle=black, opacityback=0, bottomrule=.15mm, colback=white, opacitybacktitle=0.6, title=\textcolor{quarto-callout-caution-color}{\faFire}\hspace{0.5em}{Preregistration Item}, toprule=.15mm, colframe=quarto-callout-caution-color-frame, left=2mm, leftrule=.75mm, breakable, bottomtitle=1mm, colbacktitle=quarto-callout-caution-color!10!white, toptitle=1mm, arc=.35mm]

\begin{itemize}
\tightlist
\item[$\square$]
  Describe how your conceptual model will be critically evaluated.
  Evaluation includes both the completeness and suitability of the
  overall model structure.
\item[$\square$]
  How will you critically assess any simplifying assumptions
  (\citeproc{ref-Augusiak:2014gz}{Augusiak et al., 2014})?
\item[$\square$]
  Will this process will include consultation or feedback from a client,
  manager, or model user.
\end{itemize}

\end{tcolorbox}

\section{Formalise and Specify Model}\label{formalise-and-specify-model}

\begin{tcolorbox}[enhanced jigsaw, rightrule=.15mm, titlerule=0mm, coltitle=black, opacityback=0, bottomrule=.15mm, colback=white, opacitybacktitle=0.6, title=\textcolor{quarto-callout-note-color}{\faInfo}\hspace{0.5em}{Explanation}, toprule=.15mm, colframe=quarto-callout-note-color-frame, left=2mm, leftrule=.75mm, breakable, bottomtitle=1mm, colbacktitle=quarto-callout-note-color!10!white, toptitle=1mm, arc=.35mm]

In this section describe what quantitative methods you will use to build
the model/s, explain how they are relevant to the client/manager/user's
purpose.

\end{tcolorbox}

\subsection{Model class, modelling framework and
approach}\label{model-class-modelling-framework-and-approach}

\begin{tcolorbox}[enhanced jigsaw, rightrule=.15mm, titlerule=0mm, coltitle=black, opacityback=0, bottomrule=.15mm, colback=white, opacitybacktitle=0.6, title=\textcolor{quarto-callout-note-color}{\faInfo}\hspace{0.5em}{Explanation}, toprule=.15mm, colframe=quarto-callout-note-color-frame, left=2mm, leftrule=.75mm, breakable, bottomtitle=1mm, colbacktitle=quarto-callout-note-color!10!white, toptitle=1mm, arc=.35mm]

Modelling approaches can be described as occurring on a spectrum from
correlative or phenomenological to mechanistic or process-based
(\citeproc{ref-Yates2018}{Yates et al., 2018}); where correlative models
use mathematical functions fitted to data to describe underlying
processes, and mechanistic models explicitly represent processes and
details of component parts of a biological system that are expected to
give rise to the data (\citeproc{ref-White2019a}{White \& Marshall,
2019}). A model `class,' `family'\,' or `type' is often used to describe
a set of models each of which has a distinct but related sampling
distribution (\citeproc{ref-Liu2008}{C. C. Liu \& Aitkin, 2008}). The
model family is driven by choices about the types of variables covered
and the nature of their treatment, as well as structural features of the
model, such as link functions, spatial and temporal scales of processes
and their interactions (\citeproc{ref-Jakeman:2006ii}{Jakeman et al.,
2006}).

\end{tcolorbox}

\begin{tcolorbox}[enhanced jigsaw, rightrule=.15mm, titlerule=0mm, coltitle=black, opacityback=0, bottomrule=.15mm, colback=white, opacitybacktitle=0.6, title=\textcolor{quarto-callout-caution-color}{\faFire}\hspace{0.5em}{Preregistration Item}, toprule=.15mm, colframe=quarto-callout-caution-color-frame, left=2mm, leftrule=.75mm, breakable, bottomtitle=1mm, colbacktitle=quarto-callout-caution-color!10!white, toptitle=1mm, arc=.35mm]

\begin{itemize}
\tightlist
\item[$\square$]
  Describe what modelling framework, approach or class of model you will
  use to implement your model and relate your choice to the model
  purpose and analytical objectives described in
  Section~\ref{sec-model-purpose-context-and-problem-context} and
  Section~\ref{sec-analytical-objectives}.
\end{itemize}

\end{tcolorbox}

\subsection{Choose model features and
family}\label{choose-model-features-and-family}

\begin{tcolorbox}[enhanced jigsaw, rightrule=.15mm, titlerule=0mm, coltitle=black, opacityback=0, bottomrule=.15mm, colback=white, opacitybacktitle=0.6, title=\textcolor{quarto-callout-note-color}{\faInfo}\hspace{0.5em}{Explanation}, toprule=.15mm, colframe=quarto-callout-note-color-frame, left=2mm, leftrule=.75mm, breakable, bottomtitle=1mm, colbacktitle=quarto-callout-note-color!10!white, toptitle=1mm, arc=.35mm]

All modelling approaches require the selection of model features, which
conform with the conceptual model and data specified in previous steps
(\citeproc{ref-Jakeman:2006ii}{Jakeman et al., 2006}). The choice of
model are determined in conjunction with features are selected. Model
features include elements such as the functional form of interactions,
data structures, measures used to specify links, any bins or
discretisation of continuous variables. It is usually difficult to
change fundamental features of a model beyond an early stage of model
development, so careful thought and planning here is useful to the
modeller (\citeproc{ref-Jakeman:2006ii}{Jakeman et al., 2006}). However,
if changes to these fundamental aspects of the model do need to change,
document how and why these choices were made, including any results used
to support any changes in the model.

\end{tcolorbox}

\subsubsection{Operationalising Model
Variables}\label{operationalising-model-variables}

\begin{tcolorbox}[enhanced jigsaw, rightrule=.15mm, titlerule=0mm, coltitle=black, opacityback=0, bottomrule=.15mm, colback=white, opacitybacktitle=0.6, title=\textcolor{quarto-callout-caution-color}{\faFire}\hspace{0.5em}{Preregistration Item}, toprule=.15mm, colframe=quarto-callout-caution-color-frame, left=2mm, leftrule=.75mm, breakable, bottomtitle=1mm, colbacktitle=quarto-callout-caution-color!10!white, toptitle=1mm, arc=.35mm]

\begin{itemize}
\tightlist
\item[$\square$]
  For each response, predictor, and covariate, specify how these
  variables will be operationalised in the model. This should relate
  directly to the analytical and/or management objectives specified
  during the problem formulation phase. Operationalisations could
  include: the extent of a response, an extreme value, a trend, a
  long-term mean, a probability distribution, a spatial pattern, a
  time-series, qualitative change, such as a direction of change or, the
  frequency, location, or probability of some event occuring. Specify
  any treatment of model variables, including whether they are lumped /
  distributed, lienar / non-linear, stochastic / deterministic
  (\citeproc{ref-Jakeman:2006ii}{Jakeman et al., 2006}).
\item[$\square$]
  Provide a rationale for your choices, including why plausible
  alternatives under consideration were not chosen, and relate your
  justification bacj to the purpose, objectives, prior knowledge and or
  logistical constraints specified in the problem formulation phase
  (\citeproc{ref-Jakeman:2006ii}{Jakeman et al., 2006}).
\end{itemize}

\end{tcolorbox}

\subsubsection{Choose model family}\label{choose-model-family}

\begin{tcolorbox}[enhanced jigsaw, rightrule=.15mm, titlerule=0mm, coltitle=black, opacityback=0, bottomrule=.15mm, colback=white, opacitybacktitle=0.6, title=\textcolor{quarto-callout-caution-color}{\faFire}\hspace{0.5em}{Preregistration Item}, toprule=.15mm, colframe=quarto-callout-caution-color-frame, left=2mm, leftrule=.75mm, breakable, bottomtitle=1mm, colbacktitle=quarto-callout-caution-color!10!white, toptitle=1mm, arc=.35mm]

\begin{itemize}
\tightlist
\item[$\square$]
  Specify which family of statistical distributions you will use in your
  model, and describe any transformations, or link functions.
\item[$\square$]
  Include in your rational for selection, detail about which variables
  the model outputs are likely sensitive to, what aspects of their
  behaviour are important, and any associated spatial or temporal
  dimensions in sampling.
\end{itemize}

\end{tcolorbox}

\subsection{\texorpdfstring{Describe \emph{approach} for identifying
model
structure}{Describe approach for identifying model structure}}\label{describe-approach-for-identifying-model-structure}

\begin{tcolorbox}[enhanced jigsaw, rightrule=.15mm, titlerule=0mm, coltitle=black, opacityback=0, bottomrule=.15mm, colback=white, opacitybacktitle=0.6, title=\textcolor{quarto-callout-note-color}{\faInfo}\hspace{0.5em}{Explanation}, toprule=.15mm, colframe=quarto-callout-note-color-frame, left=2mm, leftrule=.75mm, breakable, bottomtitle=1mm, colbacktitle=quarto-callout-note-color!10!white, toptitle=1mm, arc=.35mm]

This section relates to the process of determining the best/most
efficient/parsimonious representation of the system at the appropriate
scale of concern (\citeproc{ref-Jakeman:2006ii}{Jakeman et al., 2006})
that best meets the analytical objectives specified in the problem
formulation phase. Model structure refers to the choice of variables
included in the model, and the nature of the relationship among those
variables. Approaches to finding model structure and parameters may be
knowledge-supported, or data-driven (\citeproc{ref-Boets:2015gl}{Boets
et al., 2015}). Model selection methods can include traditional
inferential approaches such as unconstrained searches of a dataset for
patterns that explain variations in the response variable, or use of
ensemble-modelling methods (\citeproc{ref-Barnard2019}{Barnard et al.,
2019}). Ensemble modelling procedures might aim to derive a single
model, or a multi-model average (\citeproc{ref-Yates2018}{Yates et al.,
2018}). Refining actions to develop a model could include iteratively
dropping parameters or adding them, or aggregating / disaggregating
system descriptors, such as dimensionality and processes
(\citeproc{ref-Jakeman:2006ii}{Jakeman et al., 2006}).

\end{tcolorbox}

\begin{tcolorbox}[enhanced jigsaw, rightrule=.15mm, titlerule=0mm, coltitle=black, opacityback=0, bottomrule=.15mm, colback=white, opacitybacktitle=0.6, title=\textcolor{quarto-callout-caution-color}{\faFire}\hspace{0.5em}{Preregistration Item}, toprule=.15mm, colframe=quarto-callout-caution-color-frame, left=2mm, leftrule=.75mm, breakable, bottomtitle=1mm, colbacktitle=quarto-callout-caution-color!10!white, toptitle=1mm, arc=.35mm]

\begin{itemize}
\tightlist
\item[$\square$]
  Specify what approach and methods you will use to identify model
  structure and parameters.
\item[$\square$]
  If using a knowledge-supported approach to deriving model structure
  (either in whole or in part), specify model structural features,
  including:

  \begin{itemize}
  \tightlist
  \item
    the functional form of interactions (if any)
  \item
    data structures,
  \item
    measures used to specify links,
  \item
    any bins or discretisation of continuous variables
    (\citeproc{ref-Jakeman:2006ii}{Jakeman et al., 2006}),
  \item
    any other relevant features of the model structure.
  \end{itemize}
\end{itemize}

\end{tcolorbox}

\subsection{Describe parameter estimation technique and performance
criteria}\label{describe-parameter-estimation-technique-and-performance-criteria}

\begin{tcolorbox}[enhanced jigsaw, rightrule=.15mm, titlerule=0mm, coltitle=black, opacityback=0, bottomrule=.15mm, colback=white, opacitybacktitle=0.6, title=\textcolor{quarto-callout-note-color}{\faInfo}\hspace{0.5em}{Explanation}, toprule=.15mm, colframe=quarto-callout-note-color-frame, left=2mm, leftrule=.75mm, breakable, bottomtitle=1mm, colbacktitle=quarto-callout-note-color!10!white, toptitle=1mm, arc=.35mm]

Before calibrating the model to the data, the performance criteria for
judging the calibration (or model fit) are specified. These criteria and
their underlying assumptions should reflect the desired properties of
the parameter estimates / structure
(\citeproc{ref-Jakeman:2006ii}{Jakeman et al., 2006}). For example,
modellers might seek parameter estimates that are robust to outliers,
unbiased, and yield appropriate predictive performance. Modellers will
need to consider whether the assumptions of the estimation technique
yielding those desired properties are suited to the problem at hand. For
integrated or sub-divided models, other considerations might include
choices about where to disaggregate the model for parameter estimation;
e.g.~spatial sectioning (streams into reaches) and temporal sectioning
(piece-wise linear models) (\citeproc{ref-Jakeman:2006ii}{Jakeman et
al., 2006}).

\end{tcolorbox}

\subsubsection{Parameter estimation
technique}\label{parameter-estimation-technique}

\begin{tcolorbox}[enhanced jigsaw, rightrule=.15mm, titlerule=0mm, coltitle=black, opacityback=0, bottomrule=.15mm, colback=white, opacitybacktitle=0.6, title=\textcolor{quarto-callout-caution-color}{\faFire}\hspace{0.5em}{Preregistration Item}, toprule=.15mm, colframe=quarto-callout-caution-color-frame, left=2mm, leftrule=.75mm, breakable, bottomtitle=1mm, colbacktitle=quarto-callout-caution-color!10!white, toptitle=1mm, arc=.35mm]

\begin{itemize}
\tightlist
\item[$\square$]
  Specify what technique you will use to estimate parameter values, and
  how you will supply non-parametric variables and/or data
  (e.g.~distributed boundary conditions). For example, will you
  calibrate all variables simultaneously by optimising fit of model
  outputs to observations, or will you parameterise the model in a
  piecemeal fashion by either direct measurement, inference from
  secondary data, or some combination
  (\citeproc{ref-Jakeman:2006ii}{Jakeman et al., 2006}).
\item[$\square$]
  Identify which variables will be parameterised directly, such as by
  expert elicitation or prior knowledge.
\item[$\square$]
  Specify which algorithm(s) you will use for any data-driven parameter
  estimation, including supervised, or unsupervised machine learning,
  decision-tree, K-nearest neighbour or cluster algorithms
  (\citeproc{ref-Liu2018b}{Z. Liu et al., 2018}).
\end{itemize}

\end{tcolorbox}

\subsubsection{Parameter estimation / model fit performance
criteria}\label{sec-parameter-model-fit-performance-criteria}

\begin{tcolorbox}[enhanced jigsaw, rightrule=.15mm, titlerule=0mm, coltitle=black, opacityback=0, bottomrule=.15mm, colback=white, opacitybacktitle=0.6, title=\textcolor{quarto-callout-caution-color}{\faFire}\hspace{0.5em}{Preregistration Item}, toprule=.15mm, colframe=quarto-callout-caution-color-frame, left=2mm, leftrule=.75mm, breakable, bottomtitle=1mm, colbacktitle=quarto-callout-caution-color!10!white, toptitle=1mm, arc=.35mm]

\begin{itemize}
\tightlist
\item[$\square$]
  Specify which suite of performance criteria you will use to judge the
  performance of the model. Examples include correlation scores,
  coefficient of determination, specificity, sensitivity, AUC, etcetera
  (\citeproc{ref-Yates2018}{Yates et al., 2018}).
\item[$\square$]
  Relate any underlying assumptions of each criterion to the desired
  properties of the model, and justify the choice of performance metric
  in relation
\item[$\square$]
  Explain how you will identify which model features or components are
  significant or meaningful.
\end{itemize}

\end{tcolorbox}

\subsection{Model assumptions and
uncertainties}\label{model-assumptions-and-uncertainties-1}

\begin{tcolorbox}[enhanced jigsaw, rightrule=.15mm, titlerule=0mm, coltitle=black, opacityback=0, bottomrule=.15mm, colback=white, opacitybacktitle=0.6, title=\textcolor{quarto-callout-caution-color}{\faFire}\hspace{0.5em}{Preregistration Item}, toprule=.15mm, colframe=quarto-callout-caution-color-frame, left=2mm, leftrule=.75mm, breakable, bottomtitle=1mm, colbacktitle=quarto-callout-caution-color!10!white, toptitle=1mm, arc=.35mm]

\begin{itemize}
\tightlist
\item[$\square$]
  Specify assumptions and key uncertainties in the formal model.
  Describe what gaps exist between the model conception, and the
  real-world problem, what biases might this introduce and how might
  this impact any interpretation of the model outputs, and what
  implications are there for evaluating model-output to inform
  inferences or decisions?
\end{itemize}

\end{tcolorbox}

\subsection{Specify formal model(s)}\label{specify-formal-models}

\begin{tcolorbox}[enhanced jigsaw, rightrule=.15mm, titlerule=0mm, coltitle=black, opacityback=0, bottomrule=.15mm, colback=white, opacitybacktitle=0.6, title=\textcolor{quarto-callout-note-color}{\faInfo}\hspace{0.5em}{Explanation}, toprule=.15mm, colframe=quarto-callout-note-color-frame, left=2mm, leftrule=.75mm, breakable, bottomtitle=1mm, colbacktitle=quarto-callout-note-color!10!white, toptitle=1mm, arc=.35mm]

Once critical decisions have been made about the modelling approach and
method of model specification, the conceptual model is translated into
the quantitative model.

\end{tcolorbox}

\begin{tcolorbox}[enhanced jigsaw, rightrule=.15mm, titlerule=0mm, coltitle=black, opacityback=0, bottomrule=.15mm, colback=white, opacitybacktitle=0.6, title=\textcolor{quarto-callout-caution-color}{\faFire}\hspace{0.5em}{Preregistration Item}, toprule=.15mm, colframe=quarto-callout-caution-color-frame, left=2mm, leftrule=.75mm, breakable, bottomtitle=1mm, colbacktitle=quarto-callout-caution-color!10!white, toptitle=1mm, arc=.35mm]

\begin{itemize}
\tightlist
\item[$\square$]
  Specify all formal models

  \begin{itemize}
  \tightlist
  \item[$\square$]
    Note, For data-driven approaches to determining model structure and
    or parameterisation, it may not be possible to respond to this
    preregistration item. In such cases, explain why this is the case,
    and how you will document the model(s) used in the final analysis.
  \end{itemize}
\item[$\square$]
  For quantitative model selection approaches, including ensemble
  modelling, specify each model used in the candidate set, including any
  null or full/global model.
\end{itemize}

\end{tcolorbox}

\section{Model Calibration, Validation \&
Checking}\label{model-calibration-validation-checking}

\subsection{Model calibration and validation
scheme}\label{model-calibration-and-validation-scheme}

\begin{tcolorbox}[enhanced jigsaw, rightrule=.15mm, titlerule=0mm, coltitle=black, opacityback=0, bottomrule=.15mm, colback=white, opacitybacktitle=0.6, title=\textcolor{quarto-callout-note-color}{\faInfo}\hspace{0.5em}{Explanation}, toprule=.15mm, colframe=quarto-callout-note-color-frame, left=2mm, leftrule=.75mm, breakable, bottomtitle=1mm, colbacktitle=quarto-callout-note-color!10!white, toptitle=1mm, arc=.35mm]

This section pertains to any data calibration, validation or testing
schemes that will be implemented. For example, the model may be tested
on data independent of those used to parameterise the model (external
validation), or the model may be cross-validated on random sub-samples
of the data used to parameterise the model
(\citeproc{ref-Barnard2019}{Barnard et al., 2019}; internal
cross-validation \citeproc{ref-Yates2018}{Yates et al., 2018}). For some
types of models, hyper-parameters are estimated from data, and may be
tuned on further independent holdouts of the training data (``validation
data'').

\end{tcolorbox}

\begin{tcolorbox}[enhanced jigsaw, rightrule=.15mm, titlerule=0mm, coltitle=black, opacityback=0, bottomrule=.15mm, colback=white, opacitybacktitle=0.6, title=\textcolor{quarto-callout-caution-color}{\faFire}\hspace{0.5em}{Preregistration Item}, toprule=.15mm, colframe=quarto-callout-caution-color-frame, left=2mm, leftrule=.75mm, breakable, bottomtitle=1mm, colbacktitle=quarto-callout-caution-color!10!white, toptitle=1mm, arc=.35mm]

\begin{itemize}
\tightlist
\item[$\square$]
  Describe any data calibration, validation and testing scheme you will
  implement, including any procedures for tuning or estimating model
  hyper-parameters (if any).
\end{itemize}

\end{tcolorbox}

\subsubsection{Describe calibration/validation
data}\label{describe-calibrationvalidation-data}

\begin{tcolorbox}[enhanced jigsaw, rightrule=.15mm, titlerule=0mm, coltitle=black, opacityback=0, bottomrule=.15mm, colback=white, opacitybacktitle=0.6, title=\textcolor{quarto-callout-note-color}{\faInfo}\hspace{0.5em}{Explanation \& Rationale}, toprule=.15mm, colframe=quarto-callout-note-color-frame, left=2mm, leftrule=.75mm, breakable, bottomtitle=1mm, colbacktitle=quarto-callout-note-color!10!white, toptitle=1mm, arc=.35mm]

The following items pertain to properties of the \emph{datasets} used
for calibration (training), validation, and testing.

\end{tcolorbox}

\begin{tcolorbox}[enhanced jigsaw, rightrule=.15mm, titlerule=0mm, coltitle=black, opacityback=0, bottomrule=.15mm, colback=white, opacitybacktitle=0.6, title=\textcolor{quarto-callout-caution-color}{\faFire}\hspace{0.5em}{Preregistration Item}, toprule=.15mm, colframe=quarto-callout-caution-color-frame, left=2mm, leftrule=.75mm, breakable, bottomtitle=1mm, colbacktitle=quarto-callout-caution-color!10!white, toptitle=1mm, arc=.35mm]

\emph{If partitioning data for cross-validation or similar approach
(delete as needed):}

\begin{itemize}
\tightlist
\item[$\square$]
  Describe the approach specifying the number of folds that will be
  created, the relative size of each fold, and any stratification
  methods used for ensuring evenness of groups between folds and between
  calibration / validation data?
\end{itemize}

\emph{If using external / independent holdout data for model testing and
evaluation (delete as needed):}

\begin{itemize}
\tightlist
\item[$\square$]
  Which data will be used as a the testing data? What method will you be
  used for generating training / test data subsets?
\item[$\square$]
  Describe any known differences between the training/validation and
  testing datasets, the relative size of each, as well as any
  stratification methods used for ensuring evenness of groups between
  data sets?
\item[$\square$]
  It is preferable that any independent data used for model testing
  remains unknown to modellers during the process of model development,
  please describe the relationship modellers have to model validation
  data, will independent datasets be known or accessible to any modeller
  or analyst?
\end{itemize}

\end{tcolorbox}

\subsection{Implementation
verification}\label{implementation-verification}

\begin{tcolorbox}[enhanced jigsaw, rightrule=.15mm, titlerule=0mm, coltitle=black, opacityback=0, bottomrule=.15mm, colback=white, opacitybacktitle=0.6, title=\textcolor{quarto-callout-note-color}{\faInfo}\hspace{0.5em}{Explanation \& Examples}, toprule=.15mm, colframe=quarto-callout-note-color-frame, left=2mm, leftrule=.75mm, breakable, bottomtitle=1mm, colbacktitle=quarto-callout-note-color!10!white, toptitle=1mm, arc=.35mm]

Model implementation verification is the process of ensuring that the
model has been correctly implemented, and that the model performs as
described by the model description (\citeproc{ref-Grimm:2014es}{Grimm et
al., 2014}). This process is distinct from model checking, which
assesses the model's performance in representing the system of interest
(\citeproc{ref-Conn:2018hd}{Conn et al., 2018}).

\begin{itemize}
\tightlist
\item
  Checks for verification implementation should include i) thoroughly
  checking for bugs or programming errors, and ii) whether the
  implemented model performs as described by the model description
  (\citeproc{ref-Grimm:2014es}{Grimm et al., 2014}).
\item
  Qualitative tests could include syntax checking of code, and peer-code
  review (\citeproc{ref-ivimey2023}{Ivimey et al., 2023}). Technical
  measures include using unit tests, or in-built checks within functions
  to prevent potential errors.
\end{itemize}

\end{tcolorbox}

\begin{tcolorbox}[enhanced jigsaw, rightrule=.15mm, titlerule=0mm, coltitle=black, opacityback=0, bottomrule=.15mm, colback=white, opacitybacktitle=0.6, title=\textcolor{quarto-callout-caution-color}{\faFire}\hspace{0.5em}{Preregistration Item}, toprule=.15mm, colframe=quarto-callout-caution-color-frame, left=2mm, leftrule=.75mm, breakable, bottomtitle=1mm, colbacktitle=quarto-callout-caution-color!10!white, toptitle=1mm, arc=.35mm]

\begin{itemize}
\tightlist
\item[$\square$]
  What Quality Assurance measures will you take to verify the model has
  been correctly implemented? Specifying a priori quality assurance
  tests for implementation verification may help to avoid selective
  debugging and silent errors.
\end{itemize}

\end{tcolorbox}

\subsection{Model checking}\label{model-checking}

\begin{tcolorbox}[enhanced jigsaw, rightrule=.15mm, titlerule=0mm, coltitle=black, opacityback=0, bottomrule=.15mm, colback=white, opacitybacktitle=0.6, title=\textcolor{quarto-callout-note-color}{\faInfo}\hspace{0.5em}{Rationale \& Explanation}, toprule=.15mm, colframe=quarto-callout-note-color-frame, left=2mm, leftrule=.75mm, breakable, bottomtitle=1mm, colbacktitle=quarto-callout-note-color!10!white, toptitle=1mm, arc=.35mm]

``Model Checking'' goes by many names (``conditional verification'',
``quantitative verification'', ``model output verification'' ), and
refers to a series of analyses that assess a model's performance in
representing the system of interest (\citeproc{ref-Conn:2018hd}{Conn et
al., 2018}). Model checking aids in diagnosing assumption violations,
and reveals where a model might need to be altered to better represent
the data, and therefore system (\citeproc{ref-Conn:2018hd}{Conn et al.,
2018}). Quantitative model checking diagnostics include goodness of fit,
tests on residuals or errors, such as for heteroscedascity,
cross-correlation, and autocorrelation
(\citeproc{ref-Jakeman:2006ii}{Jakeman et al., 2006}).

\end{tcolorbox}

\subsubsection{Quantitative model
checking}\label{quantitative-model-checking}

\begin{tcolorbox}[enhanced jigsaw, rightrule=.15mm, titlerule=0mm, coltitle=black, opacityback=0, bottomrule=.15mm, colback=white, opacitybacktitle=0.6, title=\textcolor{quarto-callout-caution-color}{\faFire}\hspace{0.5em}{Preregistration Item}, toprule=.15mm, colframe=quarto-callout-caution-color-frame, left=2mm, leftrule=.75mm, breakable, bottomtitle=1mm, colbacktitle=quarto-callout-caution-color!10!white, toptitle=1mm, arc=.35mm]

\emph{During this process, observed data, or data and patterns that
guided model design and calibration, are compared to model output in
order to identify if and where there are any systematic differences.}

\begin{itemize}
\tightlist
\item[$\square$]
  Specify any diagnostics or tests you will use during model checking to
  assess a model's performance in representing the system of interest.
\item[$\square$]
  For each test, specify the criteria that will you use to interpret the
  outcome of the test in assessing the model's ability to sufficiently
  represent the gathered data used to develop and parameterise the
  model.
\end{itemize}

\end{tcolorbox}

\subsubsection{Qualitative model
checking}\label{qualitative-model-checking}

\begin{tcolorbox}[enhanced jigsaw, rightrule=.15mm, titlerule=0mm, coltitle=black, opacityback=0, bottomrule=.15mm, colback=white, opacitybacktitle=0.6, title=\textcolor{quarto-callout-note-color}{\faInfo}\hspace{0.5em}{Explanation}, toprule=.15mm, colframe=quarto-callout-note-color-frame, left=2mm, leftrule=.75mm, breakable, bottomtitle=1mm, colbacktitle=quarto-callout-note-color!10!white, toptitle=1mm, arc=.35mm]

This step is largely informal and case-specific, but requires‚ `face
validation' with model users / clients / managers who aren't involved in
the development of the model to assess whether the interactions and
outcomes of the model are feasible an defensible
(\citeproc{ref-Grimm:2014es}{Grimm et al., 2014}). This process is
sometimes called a ``laugh test'' or a ``pub test'' and in addition to
checking the model's believability, it builds the client's confidence in
the model (\citeproc{ref-Jakeman:2006ii}{Jakeman et al., 2006}). Face
validation could include structured walk-throughs, or presenting
descriptions, visualisations or summaries of model results to experts
for assessment.

\end{tcolorbox}

\begin{tcolorbox}[enhanced jigsaw, rightrule=.15mm, titlerule=0mm, coltitle=black, opacityback=0, bottomrule=.15mm, colback=white, opacitybacktitle=0.6, title=\textcolor{quarto-callout-caution-color}{\faFire}\hspace{0.5em}{Preregistration Item}, toprule=.15mm, colframe=quarto-callout-caution-color-frame, left=2mm, leftrule=.75mm, breakable, bottomtitle=1mm, colbacktitle=quarto-callout-caution-color!10!white, toptitle=1mm, arc=.35mm]

\begin{itemize}
\tightlist
\item[$\square$]
  Briefly explain how you will qualitatively check the model, and
  whether and how you will include users and clients in the process.
\end{itemize}

\end{tcolorbox}

\subsubsection{Assumption Violation
Checks}\label{assumption-violation-checks}

\begin{tcolorbox}[enhanced jigsaw, rightrule=.15mm, titlerule=0mm, coltitle=black, opacityback=0, bottomrule=.15mm, colback=white, opacitybacktitle=0.6, title=\textcolor{quarto-callout-caution-color}{\faFire}\hspace{0.5em}{Preregistration Item}, toprule=.15mm, colframe=quarto-callout-caution-color-frame, left=2mm, leftrule=.75mm, breakable, bottomtitle=1mm, colbacktitle=quarto-callout-caution-color!10!white, toptitle=1mm, arc=.35mm]

\emph{The consequences of assumption violations on the interpretation of
results should be assessed (\citeproc{ref-Araujo2019}{Araújo et al.,
2019}).}

\begin{itemize}
\tightlist
\item[$\square$]
  Explain how you will demonstrate robustness to model assumptions and
  check for violations of model assumptions.
\item[$\square$]
  If you cannot perform quantitative assumption checks, describe what
  theoretical justifications would justify a lack of violation of or
  robustness to model assumptions.
\item[$\square$]
  If you cannot demonstrate or theoretically justify violation or
  robustness to assumptions, explain why not, and specify whether you
  will discuss assumption violations and their consequences for
  interpretation of model outputs.
\item[$\square$]
  If assumption violations cannot be avoided, explain how you will
  explore the consequences of assumption violations on the
  interpretation of results (To be completed in interim iterations of
  the preregistration, only if there are departures from assumptions as
  demonstrated in the planned tests above).
\end{itemize}

\end{tcolorbox}

\section{Model Validation and
Evaluation}\label{model-validation-and-evaluation}

\begin{tcolorbox}[enhanced jigsaw, rightrule=.15mm, titlerule=0mm, coltitle=black, opacityback=0, bottomrule=.15mm, colback=white, opacitybacktitle=0.6, title=\textcolor{quarto-callout-note-color}{\faInfo}\hspace{0.5em}{Explanation}, toprule=.15mm, colframe=quarto-callout-note-color-frame, left=2mm, leftrule=.75mm, breakable, bottomtitle=1mm, colbacktitle=quarto-callout-note-color!10!white, toptitle=1mm, arc=.35mm]

The model validation \& evaluation phase comprises a suite of analyses
that collectively inform inferences about whether, and under what
conditions, a model is suitable to meet its intended purpose
(\citeproc{ref-Augusiak:2014gz}{Augusiak et al., 2014}). Errors in
design and implementation of the model and their implication on the
model output are assessed. Ideally independent data is used against the
model outputs to assess whether the model output behaviour exhibits the
required accuracy for the model's intended purpose. The outcomes of
these analyses build confidence in the model applications and increase
understanding of model strengths and limitations. Model evaluation
including, model analysis, should complement model checking. It should
evaluate model checking, and consider over-fitting and extrapolation. As
the proportion of calibrated or uncertain parameters increases, so does
the risk that the model seemingly works correctly, but for the wrong
mechanistic reasons (\citeproc{ref-Boettiger2022}{Boettiger, 2022}).
Evaluation thus complements model checking because we can rule out the
chance that the model fits the calibration data well, but has not
captured the relevant ecological mechanisms of the system pertinent to
the research question or the decision problem underpinning the model
(\citeproc{ref-Grimm:2014es}{Grimm et al., 2014}). Evaluation of model
outputs against external data in conjunction with the results from model
checking provide information about the structural realism and therefore
credibility of the model (\citeproc{ref-Grimm2016}{Grimm \& Berger,
2016}).

\end{tcolorbox}

\subsection{Model output
corroboration}\label{model-output-corroboration}

\begin{tcolorbox}[enhanced jigsaw, rightrule=.15mm, titlerule=0mm, coltitle=black, opacityback=0, bottomrule=.15mm, colback=white, opacitybacktitle=0.6, title=\textcolor{quarto-callout-note-color}{\faInfo}\hspace{0.5em}{Explanation}, toprule=.15mm, colframe=quarto-callout-note-color-frame, left=2mm, leftrule=.75mm, breakable, bottomtitle=1mm, colbacktitle=quarto-callout-note-color!10!white, toptitle=1mm, arc=.35mm]

Ideally, model outputs or predictions are compared to independent data
and patterns that were not used to develop, parameterise, or verify the
model. Testing against a dataset of response and predictor variables
that are spatially and/or temporally independent from the training
dataset minimises the risk of artificially inflating model performance
measures (\citeproc{ref-Araujo2019}{Araújo et al., 2019}). Although the
corroboration of model outputs against an independent validation dataset
is considered the `gold standard' for showing that a model properly
represents the internal organisation of the system, model validation is
not always possible because empirical experiments are infeasible or
model users are working on rapid-response time-frames, hence, why
ecologists often model in the first place
(\citeproc{ref-Grimm:2014es}{Grimm et al., 2014}). Independent
predictions might instead be tested on sub-models. Alternatively,
patterns in model output that are robust and seem characteristic of the
system can be identified and evaluated in consultation with the
literature or by experts to judge how accurate the model output is
(\citeproc{ref-Grimm:2014es}{Grimm et al., 2014}).

\end{tcolorbox}

\begin{tcolorbox}[enhanced jigsaw, rightrule=.15mm, titlerule=0mm, coltitle=black, opacityback=0, bottomrule=.15mm, colback=white, opacitybacktitle=0.6, title=\textcolor{quarto-callout-caution-color}{\faFire}\hspace{0.5em}{Preregistration Item}, toprule=.15mm, colframe=quarto-callout-caution-color-frame, left=2mm, leftrule=.75mm, breakable, bottomtitle=1mm, colbacktitle=quarto-callout-caution-color!10!white, toptitle=1mm, arc=.35mm]

\begin{itemize}
\tightlist
\item[$\square$]
  State whether you will corroborate the model outputs on external data,
  and document any independent validation data in step.
\item[$\square$]
  It is preferable that any independent data used for model evaluation
  remains unknown to modellers during the process of model building
  (\citeproc{ref-Dwork2015}{Dwork et al., 2015}), describe the
  relationship modellers have to model validation data, e.g.~will
  independent datasets be known to any modeller or analyst involved in
  the model building process?
\item[$\square$]
  If unable to evaluate the model outputs against independent data,
  explain why and explain what steps you will take to interrogate the
  model.
\end{itemize}

\end{tcolorbox}

\subsection{Choose performance metrics and
criteria}\label{choose-performance-metrics-and-criteria}

\begin{tcolorbox}[enhanced jigsaw, rightrule=.15mm, titlerule=0mm, coltitle=black, opacityback=0, bottomrule=.15mm, colback=white, opacitybacktitle=0.6, title=\textcolor{quarto-callout-note-color}{\faInfo}\hspace{0.5em}{Explanation}, toprule=.15mm, colframe=quarto-callout-note-color-frame, left=2mm, leftrule=.75mm, breakable, bottomtitle=1mm, colbacktitle=quarto-callout-note-color!10!white, toptitle=1mm, arc=.35mm]

Model performance can be quantified by a range of tests, including
measures of agreement between predictions and independent observations,
or estimates of accuracy, bias, calibration, discrimination refinement,
resolution and skill (\citeproc{ref-Araujo2019}{Araújo et al., 2019}).
Note that the performance metrics and criteria in this section are used
for evaluating the structured and parameterised models (ideally) on
independent holdout data, so this step is additional to any performance
criteria used for determining model structure or parameterisation
(Section~\ref{sec-parameter-model-fit-performance-criteria}).

\end{tcolorbox}

\begin{tcolorbox}[enhanced jigsaw, rightrule=.15mm, titlerule=0mm, coltitle=black, opacityback=0, bottomrule=.15mm, colback=white, opacitybacktitle=0.6, title=\textcolor{quarto-callout-caution-color}{\faFire}\hspace{0.5em}{Preregistration Item}, toprule=.15mm, colframe=quarto-callout-caution-color-frame, left=2mm, leftrule=.75mm, breakable, bottomtitle=1mm, colbacktitle=quarto-callout-caution-color!10!white, toptitle=1mm, arc=.35mm]

\begin{itemize}
\tightlist
\item[$\square$]
  Specify what performance measures you will use to evaluate the model
  and briefly explain how each test relates to different desired
  properties of a model's performance.
\item[$\square$]
  Spatial, temporal and environmental pattern of errors and variance can
  change the interpretation of model predictions and conservation
  decisions (\citeproc{ref-Araujo2019}{Araújo et al., 2019}), where
  relevant and possible, describe how you will characterise and report
  the spatial, temporal and environmental pattern of errors and
  variance.
\item[$\square$]
  If comparing alternative models, specify what measures of model
  comparison or out-of-sample performance metrics will you use to find
  support for alternative models or else to optimise predictive ability.
  State what numerical threshold or qualities you will use for each of
  these metrics.
\end{itemize}

\end{tcolorbox}

\subsection{Model analysis}\label{model-analysis}

\begin{tcolorbox}[enhanced jigsaw, rightrule=.15mm, titlerule=0mm, coltitle=black, opacityback=0, bottomrule=.15mm, colback=white, opacitybacktitle=0.6, title=\textcolor{quarto-callout-note-color}{\faInfo}\hspace{0.5em}{Rationale \& Explanation}, toprule=.15mm, colframe=quarto-callout-note-color-frame, left=2mm, leftrule=.75mm, breakable, bottomtitle=1mm, colbacktitle=quarto-callout-note-color!10!white, toptitle=1mm, arc=.35mm]

Uncertainty in models arises due to incomplete system understanding
(which processes to include, or which interact), from imprecise, finite
and sparese data measurements, and from uncertainty in input conditions
and scenarios for model simulations or runs
(\citeproc{ref-Jakeman:2006ii}{Jakeman et al., 2006}). Non-technical
uncertainties can also be introduced throughout the modellign process,
such as uncertainties arising from issues in problem-framing,
indeterminicies, and modeller / client values
(\citeproc{ref-Jakeman:2006ii}{Jakeman et al., 2006}).

The purpose of model analysis is to prevent blind trust in the model by
understanding how model outputs have emerged, and to `challenge' the
model by verifying whether the model is still believable and fit for
purpose if one or more parameters are changed
(\citeproc{ref-Grimm:2014es}{Grimm et al., 2014}).

Model analysis should increase understanding of the model behaviour by
identifying which processes and process interactions explain
characteristic behaviours of the model system. Model analysis typically
consists of sensitivity analyses preceded by uncertainty analyses
(\citeproc{ref-Saltelli2019}{Saltelli et al., 2019}), and a suite of
other simulation or other computational experiments. The aim of such
computational experiments is to increase understanding of the model
behaviour by identifying which processes and process interactions
explain characteristic behaviours of the model system
(\citeproc{ref-Grimm:2014es}{Grimm et al., 2014}). Uncertainty analyses
and sensitivity analyses augment one another to draw conclusions about
model uncertainty.

Because the results from a full suite of sensitivity analysis and
uncertainty analysis can be difficult to interpret due to the number and
complexity of causal relations examined
(\citeproc{ref-Jakeman:2006ii}{Jakeman et al., 2006}), it is useful for
the analyst to relate the choice of analysis to the modelling context,
purpose and analytical objectives defined in the problem formulation
phase, in tandem with any critical uncertainties that have emerged
during model development and testing prior to this point.

\end{tcolorbox}

\subsubsection{Uncertainty Analyses}\label{uncertainty-analyses}

\begin{tcolorbox}[enhanced jigsaw, rightrule=.15mm, titlerule=0mm, coltitle=black, opacityback=0, bottomrule=.15mm, colback=white, opacitybacktitle=0.6, title=\textcolor{quarto-callout-note-color}{\faInfo}\hspace{0.5em}{Explanation}, toprule=.15mm, colframe=quarto-callout-note-color-frame, left=2mm, leftrule=.75mm, breakable, bottomtitle=1mm, colbacktitle=quarto-callout-note-color!10!white, toptitle=1mm, arc=.35mm]

Uncertainty can arise from different modelling techniques, response data
and predictor variables (\citeproc{ref-Araujo2019}{Araújo et al.,
2019}). Uncertainty analyses characterise the uncertainty in model
outputs, and identify how uncertainty in model parameters affects
uncertainty in model output, but does not identify which model
assumptions are driving this behaviour
(\citeproc{ref-Grimm:2014es}{Grimm et al., 2014};
\citeproc{ref-Saltelli2019}{Saltelli et al., 2019}). Uncertainty
analyses can include propagating known uncertainties through the model,
or by investigating the effect of different model scenarios with
different parameters and modelling technique combinations
(\citeproc{ref-Araujo2019}{Araújo et al., 2019}), for example. It could
also include characterising the output distribution, such as through
empirical construction using model output data points. It could also
include extracting summary statistics like the mean, median and variance
from this distribution, and perhaps constructing confidence intervals on
the mean (\citeproc{ref-Saltelli2019}{Saltelli et al., 2019}).

\end{tcolorbox}

\begin{tcolorbox}[enhanced jigsaw, rightrule=.15mm, titlerule=0mm, coltitle=black, opacityback=0, bottomrule=.15mm, colback=white, opacitybacktitle=0.6, title=\textcolor{quarto-callout-caution-color}{\faFire}\hspace{0.5em}{Preregistration Item}, toprule=.15mm, colframe=quarto-callout-caution-color-frame, left=2mm, leftrule=.75mm, breakable, bottomtitle=1mm, colbacktitle=quarto-callout-caution-color!10!white, toptitle=1mm, arc=.35mm]

\begin{itemize}
\tightlist
\item[$\square$]
  Please describe how you will characterise model and data
  uncertainties, e.g.~propagating known uncertainties through the model,
  investigating the effect of different model scenarios with different
  parameters and modelling technique combinations
  (\citeproc{ref-Araujo2019}{Araújo et al., 2019}), or empirically
  constructing model distributions from model output data points, and
  extracting summary statistics, including the mean, median, variance,
  and constructing confidence intervals
  (\citeproc{ref-Saltelli2019}{Saltelli et al., 2019}).
\item[$\square$]
  Relate your choice of analysis to the context and purposes of the
  model described in the problem formulation phase. For instance ‚
  discrepancies between model output and observed output may be
  important for forecasting models, where cost, benefit, an risk over a
  substantial period must be gauged, but much less critical for
  decision-making or management models where the user may be satisfied
  with knowing that the predicted ranking order of impacts of
  alternative scenarios or management options is likely to be correct,
  with only a rough indication of their sizes''
  (\citeproc{ref-Jakeman:2006ii}{Jakeman et al., 2006}).
\item[$\square$]
  Briefly describe how you will summarise the results of these in silico
  experiments with graphical, tabular, or other devices, such as summary
  statistics.
\item[$\square$]
  If the chosen modelling approach is able to explicitly articulate
  uncertainty due to data, measurements or baseline conditions, such as
  by providing estimates of uncertainty (typically in the form of
  probabilistic parameter covariance,
  \citeproc{ref-Jakeman:2006ii}{Jakeman et al., 2006}), specify which
  measure of uncertainty you will use.
\end{itemize}

\end{tcolorbox}

\subsubsection{Sensitivity analyses}\label{sensitivity-analyses}

\begin{tcolorbox}[enhanced jigsaw, rightrule=.15mm, titlerule=0mm, coltitle=black, opacityback=0, bottomrule=.15mm, colback=white, opacitybacktitle=0.6, title=\textcolor{quarto-callout-note-color}{\faInfo}\hspace{0.5em}{Explanation}, toprule=.15mm, colframe=quarto-callout-note-color-frame, left=2mm, leftrule=.75mm, breakable, bottomtitle=1mm, colbacktitle=quarto-callout-note-color!10!white, toptitle=1mm, arc=.35mm]

Sensitivity analysis examines how uncertainty in model outputs can be
apportioned to different sources of uncertainty in model input
(\citeproc{ref-Saltelli2019}{Saltelli et al., 2019}).

\end{tcolorbox}

\begin{tcolorbox}[enhanced jigsaw, rightrule=.15mm, titlerule=0mm, coltitle=black, opacityback=0, bottomrule=.15mm, colback=white, opacitybacktitle=0.6, title=\textcolor{quarto-callout-caution-color}{\faFire}\hspace{0.5em}{Preregistration Item}, toprule=.15mm, colframe=quarto-callout-caution-color-frame, left=2mm, leftrule=.75mm, breakable, bottomtitle=1mm, colbacktitle=quarto-callout-caution-color!10!white, toptitle=1mm, arc=.35mm]

\begin{itemize}
\tightlist
\item[$\square$]
  Describe the sensitivity analysis approach you will take:
  deterministic sensitivity, stochastic sensitivity (variability in the
  model), or scenario sensitivity (effect of changes based on
  scenarios).
\item[$\square$]
  Describe any sensitivity analyses you will conduct by specifying which
  parameters will be held constant, which will be varied, and the range
  and intervals of values over which those parameters will be varied.
\item[$\square$]
  State the primary objective of each sensitivity analysis, for example,
  to identify which input variables contribute the most to model
  uncertainty so that these variables can be targeted for further data
  collection, or alternatively to identify which variables or factors
  contribute little to overall model outputs, and so can be `dropped'
  from future iterations of the model
  (\citeproc{ref-Saltelli2019}{Saltelli et al., 2019}).
\end{itemize}

\end{tcolorbox}

\subsubsection{Model application or scenario
analysis}\label{model-application-or-scenario-analysis}

\begin{tcolorbox}[enhanced jigsaw, rightrule=.15mm, titlerule=0mm, coltitle=black, opacityback=0, bottomrule=.15mm, colback=white, opacitybacktitle=0.6, title=\textcolor{quarto-callout-caution-color}{\faFire}\hspace{0.5em}{Preregistration Item}, toprule=.15mm, colframe=quarto-callout-caution-color-frame, left=2mm, leftrule=.75mm, breakable, bottomtitle=1mm, colbacktitle=quarto-callout-caution-color!10!white, toptitle=1mm, arc=.35mm]

\begin{itemize}
\tightlist
\item[$\square$]
  Specify any input conditions and relevant parameter values for initial
  environmental conditions and decision-variables under each scenario
  specified in Section~\ref{sec-problem-formulation}.
\item[$\square$]
  Describe any other relevant technical details of model application,
  such as methods for how you will implement any simulations or model
  projections.
\item[$\square$]
  What raw and transformed model outputs will you extract from the model
  simulations or projections, and how will you map, plot, or otherwise
  display and synthesise the results of scenario and model analyses.
\item[$\square$]
  Explain how you will analyse the outputs to answer your analytical
  objectives. For instance, describe any trade-off or robustness
  analyses you will undertake to help evaluate and choose between
  different alternatives in consultation with experts or
  decision-makers.
\end{itemize}

\end{tcolorbox}

\subsubsection{Other simulation experiments / robustness
analyses}\label{other-simulation-experiments-robustness-analyses}

\begin{tcolorbox}[enhanced jigsaw, rightrule=.15mm, titlerule=0mm, coltitle=black, opacityback=0, bottomrule=.15mm, colback=white, opacitybacktitle=0.6, title=\textcolor{quarto-callout-caution-color}{\faFire}\hspace{0.5em}{Preregistration Item}, toprule=.15mm, colframe=quarto-callout-caution-color-frame, left=2mm, leftrule=.75mm, breakable, bottomtitle=1mm, colbacktitle=quarto-callout-caution-color!10!white, toptitle=1mm, arc=.35mm]

\begin{itemize}
\tightlist
\item[$\square$]
  Describe any other simulation experiments, robustness analyses or
  other analyses you will perform on the model, including any metrics
  and their criteria / thresholds for interpreting the results of the
  analysis.
\end{itemize}

\end{tcolorbox}

\section*{References}\label{references}
\addcontentsline{toc}{section}{References}

\phantomsection\label{refs}
\begin{CSLReferences}{1}{0}
\bibitem[\citeproctext]{ref-Araujo2019}
Araújo, M., Anderson, R., Márcia Barbosa, A., Beale, C., Dormann, C.,
Early, R., Garcia, R., Guisan, A., Maiorano, L., Naimi, B., O'Hara, R.,
Zimmermann, N., \& Rahbek, C. (2019). Standards for distribution models
in biodiversity assessments. \emph{Sci Adv}, \emph{5}(1), eaat4858.
\url{https://doi.org/10.1126/sciadv.aat4858}

\bibitem[\citeproctext]{ref-Augusiak:2014gz}
Augusiak, J., Van den Brink, P. J., \& Grimm, V. (2014). Merging
validation and evaluation of ecological models to {``evaludation''}: A
review of terminology and a practical approach. \emph{Ecological
Modelling}, \emph{280}, 117--128.
\url{https://doi.org/10.1016/j.ecolmodel.2013.11.009}

\bibitem[\citeproctext]{ref-Barnard2019}
Barnard, D. M., Germino, M. J., Pilliod, D. S., Arkle, R. S.,
Applestein, C., Davidson, B. E., \& Fisk, M. R. (2019). Can't see the
random forest for the decision trees: Selecting predictive models for
restoration ecology. \emph{Restoration Ecology}.
\url{https://doi.org/10.1111/rec.12938}

\bibitem[\citeproctext]{ref-Boets:2015gl}
Boets, P., Landuyt, D., Everaert, G., Broekx, S., \& Goethals, P. L. M.
(2015). \emph{Evaluation and comparison of data-driven and
knowledge-supported bayesian belief networks to assess the habitat
suitability for alien macroinvertebrates}. \emph{74}, 92--103.
\url{https://doi.org/10.1016/j.envsoft.2015.09.005}

\bibitem[\citeproctext]{ref-Boettiger2022}
Boettiger, C. (2022). The forecast trap. \emph{Ecol Lett}.
\url{https://doi.org/10.1111/ele.14024}

\bibitem[\citeproctext]{ref-Cartwright:2016kr}
Cartwright, S. J., Bowgen, K. M., Collop, C., Hyder, K., Nabe-Nielsen,
J., Stafford, R., Stillman, R. A., Thorpe, R. B., \& Sibly, R. M.
(2016). Communicating complex ecological models to non-scientist end
users. \emph{Ecological Modelling}, \emph{338}, 51--59.
\url{https://doi.org/10.1016/j.ecolmodel.2016.07.012}

\bibitem[\citeproctext]{ref-Conn:2018hd}
Conn, P. B., Johnson, D. S., Williams, P. J., Melin, S. R., \& Hooten,
M. B. (2018). A guide to bayesian model checking for ecologists.
\emph{Ecological Monographs}, \emph{9}, 341--317.
\url{https://doi.org/10.1002/ecm.1314}

\bibitem[\citeproctext]{ref-Dwork2015}
Dwork, C., Feldman, V., Hardt, M., Pitassi, T., Reingold, O., \& Roth,
A. (2015). The reusable holdout: Preserving validity in adaptive data
analysis. \emph{Science}, \emph{349}(6248), 636--638.
\url{https://doi.org/10.1126/science.aaa9375}

\bibitem[\citeproctext]{ref-Fraser:2017jf}
Fraser, H., Rumpff, L., Yen, J. D. L., Robinson, D., \& Wintle, B. A.
(2017). Integrated models to support multiobjective ecological
restoration decisions. \emph{Conservation Biology}, \emph{31}(6),
1418--1427. \url{https://doi.org/10.1111/cobi.12939}

\bibitem[\citeproctext]{ref-Gould2024a}
Gould, E., Jones, C., Yen, J. D. L., Fraser, H., Wootton, H., Vivian,
L., Good, M., Duncan, D., Rumpff, L., \& Fidler, F. (2024).
\emph{EcoConsPreReg: A guide to adaptive preregistration for model-based
research in ecology and conservation}.
\url{https://doi.org/10.5281/ZENODO.10807029}

\bibitem[\citeproctext]{ref-Grimm:2014es}
Grimm, V., Augusiak, J., Focks, A., Frank, B. M., Gabsi, F., Johnston,
A. S. A., Liu, C., Martin, B. T., Meli, M., Radchuk, V., Thorbek, P., \&
Railsback, S. F. (2014). Towards better modelling and decision support:
Documenting model development, testing, and analysis using TRACE.
\emph{Ecological Modelling}, \emph{280}, 129--139.
\url{https://doi.org/10.1016/j.ecolmodel.2014.01.018}

\bibitem[\citeproctext]{ref-Grimm2016}
Grimm, V., \& Berger, U. (2016). Structural realism, emergence, and
predictions in next-generation ecological modelling: Synthesis from a
special issue. \emph{Ecological Modelling}, \emph{326}, 177--187.
\url{https://doi.org/10.1016/j.ecolmodel.2016.01.001}

\bibitem[\citeproctext]{ref-ivimey2023}
Ivimey, E. R., Pick, J. L., Bairos, K. R., Culina, A., Gould, E.,
Grainger, M., Marshall, B. M., Moreau, D., Paquet, M., Royauté, R.,
Sánchez, A., Silva, I., \& Windecker, S. M. (2023). Implementing code
review in the scientific workflow: Insights from ecology and
evolutionary biology. \emph{Journal of Evolutionary Biology}.
https://doi.org/\url{https://doi.org/10.1111/jeb.14230}

\bibitem[\citeproctext]{ref-Jakeman:2006ii}
Jakeman, A. J., Letcher, R. A., \& Norton, J. P. (2006). Ten iterative
steps in development and evaluation of environmental models.
\emph{Environmental Modelling \& Software}, \emph{21}(5), 602--614.
\url{https://doi.org/10.1016/j.envsoft.2006.01.004}

\bibitem[\citeproctext]{ref-Liu2008}
Liu, C. C., \& Aitkin, M. (2008). Bayes factors: Prior sensitivity and
model generalizability. \emph{Journal of Mathematical Psychology},
\emph{52}(6), 362--375. \url{https://doi.org/10.1016/j.jmp.2008.03.002}

\bibitem[\citeproctext]{ref-Liu2018b}
Liu, Z., Peng, C., Work, T., Candau, J.-N., DesRochers, A., \& Kneeshaw,
D. (2018). Application of machine-learning methods in forest ecology:
Recent progress and future challenges. \emph{Environmental Reviews},
\emph{26}(4), 339--350.
https://doi.org/\url{https://doi.org/10.1139/er-2018-0034}

\bibitem[\citeproctext]{ref-Mahmoud2009}
Mahmoud, M., Liu, Y., Hartmann, H., Stewart, S., Wagener, T., Semmens,
D., Stewart, R., Gupta, H., Dominguez, D., Dominguez, F., Hulse, D.,
Letcher, R., Rashleigh, B., Smith, C., Street, R., Ticehurst, J., Twery,
M., Delden, H. van, Waldick, R., \ldots{} Winter, L. (2009). A formal
framework for scenario development in support of environmental
decision-making. \emph{Environmental Modelling \& Software},
\emph{24}(7), 798--808.
\url{https://doi.org/10.1016/j.envsoft.2008.11.010}

\bibitem[\citeproctext]{ref-McCarthy2011}
McCarthy, M. A., Thompson, C. J., Moore, A. L., \& Possingham, H. P.
(2011). Designing nature reserves in the face of uncertainty.
\emph{Ecology Letters}, \emph{14}(5), 470--475.
\url{https://doi.org/10.1111/j.1461-0248.2011.01608.x}

\bibitem[\citeproctext]{ref-McDonald-Madden2008}
McDonald-Madden, E., Baxter, P. W. J., \& Possingham, H. P. (2008).
Making robust decisions for conservation with restricted money and
knowledge. \emph{Journal of Applied Ecology}, \emph{45}(6), 1630--1638.
\url{https://doi.org/10.1111/j.1365-2664.2008.01553.x}

\bibitem[\citeproctext]{ref-Moallemi2019}
Moallemi, E. A., Elsawah, S., \& Ryan, M. J. (2019). Strengthening
{``good''} modelling practices in robust decision support: A reporting
guideline for combining multiple model-based methods. \emph{Mathematics
and Computers in Simulation}.
\url{https://doi.org/10.1016/j.matcom.2019.05.002}

\bibitem[\citeproctext]{ref-Moon2019}
Moon, K., Guerrero, A. M., Adams, Vanessa. M., Biggs, D., Blackman, D.
A., Craven, L., Dickinson, H., \& Ross, H. (2019). Mental models for
conservation research and practice. \emph{Conservation Letters},
\emph{12}(3), e12642. \url{https://doi.org/10.1111/conl.12642}

\bibitem[\citeproctext]{ref-Saltelli2019}
Saltelli, A., Aleksankina, K., Becker, W., Fennell, P., Ferretti, F.,
Holst, N., Li, S., \& Wu, Q. (2019). Why so many published sensitivity
analyses are false: A systematic review of sensitivity analysis
practices. \emph{Environmental Modelling \& Software}, \emph{114},
29--39. \url{https://doi.org/10.1016/j.envsoft.2019.01.012}

\bibitem[\citeproctext]{ref-White2019a}
White, C. R., \& Marshall, D. J. (2019). Should we care if models are
phenomenological or mechanistic. \emph{Trends in Ecology \& Evolution},
\emph{34}(4), 276--278. \url{https://doi.org/10.1016/j.tree.2019.01.006}

\bibitem[\citeproctext]{ref-Yates2018}
Yates, K., Bouchet, P., Caley, M., Mengersen, K., Randin, C., Parnell,
S., Fielding, A., Bamford, A., Ban, S., Barbosa, A., Dormann, C., Elith,
J., Embling, C., Ervin, G., Fisher, R., Gould, S., Graf, R., Gregr, E.,
Halpin, P., \ldots{} Sequeira, A. (2018). Outstanding challenges in the
transferability of ecological models. \emph{Trends Ecol. Evol. (Amst.)},
\emph{33}(10), 790--802.
\url{https://doi.org/10.1016/j.tree.2018.08.001}

\end{CSLReferences}




\end{document}
